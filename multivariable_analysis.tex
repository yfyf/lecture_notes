\documentclass[12pt]{article}
\usepackage{amsfonts, amsthm, amsmath}
\usepackage{verbatim}
\usepackage{enumerate}
\usepackage{graphicx}
\usepackage{amssymb}
\usepackage{epstopdf}
\usepackage{amsthm}
\usepackage{pb-diagram}



\setlength{\textwidth}{6.5in}
\setlength{\oddsidemargin}{0in}
\setlength{\textheight}{9.5in}
\setlength{\topmargin}{0in}
\setlength{\headheight}{0in}
\setlength{\headsep}{0in}
\setlength{\parskip}{0pt}
\setlength{\parindent}{0pt}



\def\CC{\mathbb{C}}
\def\FF{\mathbb{F}}
\def\PP{\mathbb{P}}
\def\QQ{\mathbb{Q}}
\def\RR{\mathbb{R}}
\def\ZZ{\mathbb{Z}}
\def\gotha{\mathfrak{a}}
\def\gothb{\mathfrak{b}}
\def\gothm{\mathfrak{m}}
\def\gotho{\mathfrak{o}}
\def\gothp{\mathfrak{p}}
\def\gothq{\mathfrak{q}}
\DeclareMathOperator{\disc}{Disc}
\DeclareMathOperator{\Gal}{Gal}
\DeclareMathOperator{\GL}{GL}
\DeclareMathOperator{\Hom}{Hom}
\DeclareMathOperator{\Norm}{Norm}
\DeclareMathOperator{\Trace}{Trace}
\DeclareMathOperator{\Cl}{Cl}

\def\head#1{\medskip \noindent \textbf{#1}.}

\newtheorem{theorem}{Theorem}[section]
\newtheorem{lemma}[theorem]{Lemma}
\newtheorem{definition}{Definition}[section]
\newtheorem{example}{Example}[section]
\newtheorem{proposition}{Proposition}[section]
\newtheorem{corollary}{Corollary}[section]
\newtheorem*{note}{Note}
\newtheorem*{remark}{Remark}
\newtheorem*{claim}{Claim}



\begin{document}
\title{Multivariate Analysis}
\author{Iannis Petridis}
\date{Autum 2011}
\maketitle

\tableofcontents
\setcounter{tocdepth}{4}
\newpage

\section{Mulivarible  Calculus}
\subsection{Notation}
$X\in\RR^{n}$, $ X=\{x_{1},x_{2},\dots,x_{n}\}$ where $x_{i}\in\RR$
$\RR^{n}$ is a vector space\\
length norm$|x|=\sqrt{x_1^2 + x_2^2 + \dots +x_n^2 }$\\
If $Y,X\in\RR^{n}$ and $ Y=\{y_{1},y_{2},\dots,y_{n}\}$ then $X \cdot Y=x_{1}y_{1}+ x_{2}y_{2} +\dots + x_{n}y_{n}$\\
Standard Basis:
\begin{align*}
  e_j=(0, \dots &,  0, 1, 0, \dots)\\
& \textrm{ \scriptsize j-1,\: j, \:j+1} 
\end{align*}
Properties of norm
\[|x|\geq0\]
\[|x|=0 iff x=\vec{0}\]


\subsection{linear Transformation}
\[T:\RR^{n}\rightarrow\RR^{n}\]
\begin{enumerate}[(i)]
\item $T(x+y)=T(x) + T(y)$
\item $ T(\lambda x) =\lambda T(x) $
\end{enumerate}
Matrix Representation of T with respect to the standard basis:\\
\[T(e_i)=\sum_{j=1}^{m}a_{i,j}e_j \textrm{ where } [T]_{\epsilon}^{\epsilon}=A=(a_i,j)_{\substack{i=1,\dots ,m \\ j=1,\dots ,n}}\]

Given: $T:\RR^{n}\rightarrow\RR^{m}, S:\RR^{n}\rightarrow\RR^{m} \textrm{ and } U:\RR^{m}\rightarrow\RR^{k}$
\begin{enumerate}[(i)]
\item $ [UT]_{kxm}=[U]_{kxm}[T]_{mxn}$
\item $[T+S]=[T]+[S]$
\item $\lambda[T]=[\lambda T]$
\end{enumerate}
$T:\RR^{n}\rightarrow\RR^{m}, X\in \RR^{n}, Y\in \RR^{m}, X=(x^{1},\dots ,x^{n}), Y=(y^{1},\dots ,y^{m})$\\
\[  \left(\! \begin{array}{c} y^{1} \\ y^{2}\\ \vdots\\ y^{m} \end{array}\! \right) = [T] \left(\! \begin{array}{c} x^{1} \\ x^{2}\\ \vdots\\ x^{n} \end{array}\! \right) \]

\subsection{Functions \& Continuity}
\[f:\RR^{n} \rightarrow\RR^{m}\] vector valued function
\[f:A \rightarrow\RR^{m}\] where $A \subset \RR^{n}$\\
$f$ has components which are scalar fields\\
$ f^{i}:A \rightarrow\RR$\\
\[f(x)=(f^{1} (x),\dots ,f^{m}(x))\] 

$\Pi^{i}:\RR^{m}\rightarrow\RR $
\[\Pi^{i}((x)^{1},\dots ,(x)^{m})\]
$\Pi^{i}$ is a linear transformation for i=1,$\dots$,m\\

\[
\begin{diagram}
\node{\RR^{n}} \arrow{e,t}{f}  \arrow{se,b}{f^{i}}
\node{\RR^m}  \arrow{s,r}{\Pi^{i}} \\
 \node[2]{\RR}
\end{diagram}
\]

\begin{definition}
$f:\RR^{n} \rightarrow \RR^{m}$ then 
$\lim_{x\to a} (f(x))=b$ means:
\[
\forall \epsilon > 0, \exists \delta > 0 \; st,\; 
0<|x-a|<\delta \implies |f(x)-b|<\epsilon\] 
\end{definition}

\begin{definition}
$f$ is called continuious at a if:
\[\lim_{x\to a} (f(x))=f(a)\]
$f$ is called continuious at the set of A if it is continuious at a $\forall a \in A$l
\end{definition}

\begin{theorem}[Combination Theorm]\label{T:Combination Theorm}
Assume \[\lim_{x\to a} (f(x))=b, lim_{x\to a} (g(x))=c\]
then:
\begin{enumerate}[(i)]
\item$\lim_{x\to a} (f(x) + g(x))=b+c$
\item$\lim_{x\to a} (\lambda f(x))=\lambda b$
\item$\lim_{x\to a} (f(x)\cdot g(x))=b\cdot c$
\item$\lim_{x\to a} |f(x)|=|b|$
\end{enumerate}
\begin{proof}
of (iii)
\begin{align*}
f(x)\cdot g(x)-b\cdot c &= f(x)\cdot g(x) -b\cdot g(x) + b\cdot g(x) -b\cdot c \\
&= g(x)\dot (f(x)-b) + b\cdot (g(x) - c) \\
|f(x)\cdot g(x)-b\cdot c|&= |g(x)\dot (f(x)-b) + b\cdot (g(x) - c)|\\
 &\leq |g(x)\dot (f(x)-b)| +| b\cdot (g(x) - c)|\\
\end{align*}
Cauchy-Schwartz: $|x^{1}y^{1} + \dots +x^{n}y^{n}| \leq \sqrt{(x^{1})^{2} + \dots +(x^{n})^{2}} \cdot \sqrt{(y^{1})^{2} + \dots +(y^{n})^{2}}$
\[|f(x)\cdot g(x)-b\cdot c| \leq |g(x)\dot (f(x)-b)| +| b\cdot (g(x) - c)|  \leq |g(x)|\cdot |f(x)-b| +|b|\cdot |g(x) - c|\]
Since $ lim_{x\to a} (g(x))=c$, $g$ is a bounded neighbourhood of a, i.e: 
\[
\forall M \leq 0, \exists \delta > 0 \; st,\; 
|g(x)| \leq M for |x-a|<\delta\] 
\end{proof}
\end{theorem}

\begin{remark} We have:\\
\begin{enumerate}[(i)]
\item $f:\RR^{n} \rightarrow \RR^{m}$ is continuious iff: $f^{i}:\RR^{n} \rightarrow \RR$ is continuious for $i=1,\dots , m$
\item Polynomial functions in n variables, $f(x^{1}, \dots ,x^{n})$, are continuious
\item Rational functions, $R(x)= \frac{P(x)}{Q(x)}$, are continuious where defined, ie: $Q(x) \neq 0$ and P, Q are polynomials in n variables.
\end{enumerate}
\end{remark}

\begin{theorem}\label{T:Lin trans cont}
Linear transformations are continuious.
\begin{proof}
$T:\RR^{n} \rightarrow \RR^{m}$ let $a \in \RR^{n}$ to show: $\lim_{x\to a}T(a+h) = T(a)$
\begin{align*}
|T(a+h) - T(a)| & = |T(h)|=|T(h^{1}e_{1}+ \dots +h^{n}e_{n})| =|h^{1}T(e_{1}) + \dots +h^{n}T(e_{n})|\\ 
&\leq  |h^{1}||T(e_{1})|+ \dots |h^{n}||T(e_{n})| \leq |h|(T(e_{1})+ \dots T(e_{n})) \\
So: \quad |T(a+h) - T(a)| &\leq  M|h| \quad where \quad M= \sum_{i=1}^{n}|T(e_i)| \\
So \: given \quad \epsilon > 0,\quad choose \quad \delta &= \frac{\epsilon}{M} \quad such \: that \quad |h|< \delta \implies |T(a+h) - T(a)|< \epsilon \\
\end{align*}
\end{proof}
\end{theorem}

\begin{example}
$f(x,y)= \frac{x^{2} - y^{2}}{x^{2} +y^{2}}, \quad (x,y)=(0,0)$ assume $\quad \lim_{(x,y) \to (0,0)} f(x,y) = L$
\begin{align*}
\forall \epsilon > 0, \quad  \exists \delta >0  \quad &such\: that \quad 0<|(x,y)|<\delta \implies |f(x,y)-L|<\epsilon \\
\text{Plug (x,0) into f:} f(x,0) &\;= \frac{x^{2}-0}{x^{2}-0} = 1\\
\text{Plug (0,y) into f:}  f(0,y)&\;= \frac{0-y^{2}}{0 +y^{2}} = -1\\
If \: |x|< \delta \quad |f(x,0)| < \delta & \implies |f(x,0) - L|< \epsilon \quad ie \quad |1-L|< \epsilon\\
If \: |y|< \delta \quad |f(0,y)| < \delta& \implies |f(0,y) - L|< \epsilon \quad ie \quad |-1-L|< \epsilon\\
&\implies \epsilon = \frac{1}{2} \quad contradiction! \\
\text{Now consider} \quad y=mx, m\in \RR&\\
&f(x,mx)=\frac{x^{2} - (mx)^{2}}{x^{2} + (mx)^{2}} = \frac{1-m^{2}}{1+m^{2}}\\
&\lim_{x \to 0}(\lim_{y \to 0} f(x,y)) = \lim_{x \to 0}1 = 1\\
&\lim_{y \to 0}(\lim_{x \to 0} f(x,y)) = \lim_{y \to 0}-1 = -1\\
\end{align*}
\text{However checking along straight lines is  not enough to prove continuity.}\\

\end{example}

\begin{example}

\[
 f(x,y) =
  \begin{cases}
   \frac{xy}{\sqrt{x^{2} + y^{2}}} & \text{if } (x,y) \neq 90,0) \\
   0       & \text{if } 9x,y) = (0,0)
  \end{cases}
\]
Show f is continuious at (0,0)
\[\forall \epsilon > 0, \quad \exists \delta>0\]
\[\left| \frac{xy}{\sqrt{x^{2} + y^{2}}}\right| \leq \frac{|x| \cdot |y|}{\sqrt{x^{2} + y^{2}}} \leq \frac{\sqrt{x^{2} + y^{2}} \cdot \sqrt{x^{2} + y^{2}}}{\sqrt{x^{2} + y^{2}}}= \sqrt{x^{2} + y^{2}} = |(x,y)|\]
Since: \[\begin{diagram}
\node[3]{.} \arrow{s,r,-}{y} \arrow{wsw,t,-}{ \sqrt{x^{2} + y^{2}}} \\
\node{.} \arrow[2]{e,b,-}{x} \node[2]{.}
\end{diagram}\]
\end{example}
\begin{note}
if the total degree of the neumerator is higher than the denominator in a rational function. Then the limit should be 0.
\end{note}

\begin{theorem}\label{T:composition}
If $f$ is continuious at a and $g$ is continuious at $f(a)$ then $g \circ f$ is continuious at a.
\end{theorem}

\subsection{Partial Derivitives}

\begin{definition}
Let $f:\RR^{n} \rightarrow \RR, \: a\in \RR$
\[Define: \quad D_{i}f(a) = \lim_{h \to 0}\frac{f(a^{1}, \dots , a^{i-1}, a^{i}+h, a^{i+1}, \dots , a^{n})}{h}\]
\end{definition}

\begin{example}
if $f:\RR^{n} \rightarrow \RR$
\[\left.\frac{df}{dx}\right| _{(a,b)} = D_{1}f(a,b)\]
\[\left.\frac{df}{dy}\right| _{(a,b)} = D_{2}f(a,b)\]
\[and \: in \: \RR^{3} \: we\: use \: \frac{df}{dx}, \frac{df}{dy} \: and \: \frac{df}{dz} \: etc.\]
\end{example}

\begin{example}
\begin{align*}
 f(x,y) &=
  \begin{cases}
   \frac{x^{2} - y^{2}}{\sqrt{x^{2} + y^{2}}} & \text{if } (x,y) \neq (0,0) \\
   1       & \text{if } (x,y) = (0,0)
  \end{cases}\\
 D_{1}f(0,0) =\left.\frac{df}{dx}\right| _{(0,0)} &=  \lim_{x \to 0}\frac{f(x,0) - f(0,0)}{x}= \lim_{x \to 0}\frac{\frac{x^{2}-0}{x^{2}-0} -1}{x} = 0\\
D_{2}f(0,0) =\left.\frac{df}{dy}\right| _{(0,0)} &=  \lim_{y \to 0}\frac{f(0,y) - f(0,0)}{y}= \lim_{y \to 0}\frac{\frac{0-y^{2}}{0+y^{2}} -1}{y} = \frac{-2}{y}= \pm \infty\\
\end{align*}
\end{example}

\subsection{Total Derivitive}


In 1 dimention we write the following for the derivitive of $f:\RR \rightarrow \RR$
\[\quad f^{'}(a)=\lim_{h \to 0}\frac{f(a+h) - f(a)}{h}\]
we try to write it in higher dimentions $f:\RR^{n} \rightarrow \RR^{m}$ in this form
\begin{align*}
\lim_{h \to 0}\left[\frac{f(a+h) - f(a)}{h} - f^{'}(a)\right] &=\lim_{h \to 0}\left[\frac{f(a+h) - f(a) - h \cdot f^{'}(a)}{h}\right]\\
&=\lim_{h \to 0}\frac{|f(a+h) - f(a) - h \cdot f^{'}(a)|}{|h|} =0 \\
\end{align*}
For $f:\RR^{n} \rightarrow \RR^{m}$ consider the tangent line at a: $y=f(a) +f^{'}(a)(x-a)$\\
call $x-a = h$ then we have: $y=f(a) +f^{'}(a)(h)$\\
this is an Affine transformation, not a linear map.\\
Look at the map:
\[\lambda:h \rightarrow hf^{'}(a), \quad h \in \RR\]
This is a linear map.
\begin{align*}
\lambda(h_{1} + h_{2})&=(h_{1} + h_{2})f^{'}(a)= h_{1}f^{'}(a) + h_{2}f^{'}(a) =\lambda(h_{1}) + \lambda(h_{2})\\
\lambda(\alpha \cdot h)&=(\alpha h)f^{'}(a)=\alpha(hf^{'}(a))=\alpha \cdot\lambda( h)\\
&\lim_{h \to 0}\frac{|f(a+h) - f(a) - \lambda(h)|}{|h|} =0 \\
\end{align*}

\begin{definition}[Total Derivitive]\label{D:Total Derivitive}
$f:\RR^{n} \rightarrow \RR^{m}\: or \:(f:A \rightarrow \RR^{m}, \: A \subset \RR^{n}, \:A\;is\;open)$
is differentiable at a $(a \in A)$ if we can rind a linear transformation $ \lambda:\RR^{n} \rightarrow \RR^{m}$ st:
\[\lim_{h \to 0}\frac{|f(a+h) - f(a) - \lambda(h)|}{|h|} =0 \]
The linear transformation $\lambda$ is called the total derivitive of f at a and denoted Df(a) st
\[Df(a)=\lambda(h)\] 
\end{definition}

\begin{example} 
$f:\RR^{n} \rightarrow \RR^{m}, \: f(x)=k, \: k \in\RR^{m} $ is differentiable at $a\in\RR^{n}$ with the 0 linear transformation $0:f:\RR^{n} \rightarrow \RR^{m}, \: 0(h)=0$
\[\lim_{h \to 0}\frac{|f(a+h) - f(a) - 0(h)|}{|h|} = \lim_{h \to 0}\frac{|k - k - 0|}{|h|}= 0 \]
\end{example}

\begin{example}
If $f:\RR^{n} \rightarrow \RR^{m}$ is a linear transformaton, it is differentiable at $a\in\RR^{n}$ with linear transformation $Df(a) = f$
\[\lim_{h \to 0}\frac{|f(a+h) - f(a) - f(h)|}{|h|}= \lim_{h \to 0}\frac{|f(a+h -a -h)|}{|h|} = 0\]
\end{example}

\begin{theorem}[Uniqueness of Total Derivitive]\label{T:Uniqueness of Total Derivitive}
If f is differentiable at a then there exists a unique linear transformation, $ \lambda:\RR^{n} \rightarrow \RR^{m}$, such that
\[\lim_{h \to 0}\frac{|f(a+h) - f(a) - \lambda(h)|}{|h|} =0 \]
\begin{proof}
suppose $ \mu:\RR^{n} \rightarrow \RR^{m}$ is another linear transformation such that:
\[\lim_{h \to 0}\frac{|f(a+h) - f(a) - \mu(h)|}{|h|} =0 \]
deduce that $\lambda= \mu \:\forall h\in\RR^{n} \; ie\; \lambda(h)= \mu(h)$
\begin{align*}
\frac{|\lambda(h)- \mu(h)|}{|h|}&= \frac{|\lambda(h) +f(a) -f(a+h) +f(a+h) -f(a)- \mu(h)|}{|h|}\\
&\leq \frac{|f(a+h) -f(a) -\lambda(h)|}{|h|} +  \frac{|f(a+h) -f(a) -\mu(h)|}{|h|}\\
\text{Conclude that:} \qquad \qquad \qquad &\\
\:lim_{h \to 0}\frac{|\lambda(h)- \mu(h)|}{|h|} &\leq 0 + 0 =0 \quad (*)
\end{align*}
Let h=0 $\lambda= 0=\mu$ since $\lambda, \mu$ are linear. Now fix $h \in \RR^{n}$, $h\neq0$ and let $t\in\RR$ such that $th\in\RR^{n} $ then replace $h$ with $th$ in (*):
\begin{align*}
\lim_{t \to 0}\frac{|\lambda(th)- \mu(th)|}{|th|}&= \lim_{t \to 0}\frac{|t\lambda(h)- t\mu(h)|}{|t||h|}\\
&= \lim_{t \to 0}\frac{|t|}{|t|}\frac{|\lambda(h)- \mu(h)|}{|h|}= \frac{|\lambda(h)- \mu(h)|}{|h|} = 0\\
&So\quad \lambda(h)=\mu(h)\\
\end{align*}
\end{proof}
\end{theorem}

\begin{definition}[Jacobian Matrix]\label{D:Jacobian Matrix}
$f:\RR^{n} \rightarrow \RR^{m}$ is differentiable at $a \in \RR^{n}$ and it is derivitive at a $Df(a):\RR^{n} \rightarrow \RR^{m}$ is a linear map. Then the matrix representation of $Df(a)$ is $f^{'}(a) \in |MM_{mxn}$ and is called the Jacobian Matrix of f at a.
\end{definition}

\begin{example}\label{Df example}
$f:\RR^{2} \rightarrow \RR^{2}, \quad f(x,y)=(x^{2},x+5) \quad x,y \in \RR$\\
Show that $Df(1,2)(h^{1}, h^{2})=(4h^{1} +  h^{2}, h^{1})$:
\begin{align*}
&f((1,2) +(h^{1}, h^{2})) - f(1,2) -Df(1,2)(h^{1}, h^{2})\\
&= f(1+h^{1}, 2+ h^{2}) - f(1,2) - (4h^{1}+ h^{2}, h^{1})\\
&=((1+h^{1})^{2}(2+ h^{2}), (1+h^{1} +5)) - (2,6)  - (4h^{1}+ h^{2}, h^{1})\\
&=(2+ h^{2} + 2(h^{1})^{2} +(h^{1})^{2}h^{2} + 2h^{1}h^{2} + 4h^{1} -2 -4h^{1} -h^{2}, 6+h^{1} - 6 -h^{1})
\end{align*}
\textrm{Take length:}
\[|(2(h^{1})^{2} +(h^{1})^{2}h^{2} + 2h^{1}h^{2}, 0)| \leq 2|h|^{2} +|h|^{2}|h| + 2|h||h| = 4|h|^{2} + |h|^{3}\]
\textrm{So:}
\begin{align*}
&\lim_{h \to 0}\frac{|f((1,2) +(h^{1}, h^{2})) - f(1,2) -Df(1,2)(h^{1}, h^{2})|}{|h|}\\
& \leq
\lim_{h \to 0}\frac{4|h|^{2} + |h|^{3}}{|h|} =  \lim_{h \to 0}4|h| + |h|^{2}=0
\end{align*}
\end{example}

\begin{definition}
$f^{'}(a)$ is the matrix representation of $Df(a)$
\[Df(a)(h)^{t}= \left(\! \begin{array}{c} y^{1} \\ y^{2}\\ \vdots\\ y^{m} \end{array}\! \right) = f^{'}(a) \left(\! \begin{array}{c} h^{1} \\ h^{2}\\ \vdots\\ h^{n} \end{array}\! \right) \]
\[ f^{'}(a) = \begin{pmatrix}
  D_{1}f^{1}(a) & D_{2}f^{1}(a) & \cdots &D_{n}f^{1}(a) \\
  D_{1}f^{2}(a) & D_{2}f^{2}(a) & \cdots & D_{n}f^{2}(a) \\
  \vdots  & \vdots  & \ddots & \vdots  \\
  D_{1}f^{m}(a) & D_{2}f^{m}(a) & \cdots & D_{n}f^{m}(a)
 \end{pmatrix}\]
\end{definition}

\begin{example} With this new information we can tackle example~\ref{Df example}:\\
$f:\RR^{2} \rightarrow \RR^{2}, \quad f(x,y)=(x^{2},x+5) \quad x,y \in \RR$\\
Show that $Df(1,2)(h^{1}, h^{2})=(4h^{1} +  h^{2}, h^{1})$:
\[\frac{df^{1}}{dx} = 2xy, \quad \frac{df^{1}}{dy} = x^{2}, \quad \frac{df^{2}}{dx} = 1, \quad \frac{df^{2}}{dy} = 0\]
\[f^{'}(1,2)=\begin{pmatrix}
 4&1 \\
  1&0 \\
 \end{pmatrix}\]
\[f^{'}(1,2) \left(\!\! \begin{array}{c} h^{1} \\  h^{2} \end{array}\!\! \right)= \begin{pmatrix}
 4&1 \\
  1&0 \\
 \end{pmatrix}\!\!\! \left(\!\! \begin{array}{c} h^{1} \\  h^{2} \end{array}\!\! \right)=  \left(\!\!\! \begin{array}{c} 4h^{1} + h^{2}\\  h^{2} \end{array}\!\! \right)\]
\end{example}

\begin{remark}
Having  directional derivitives in all directions $u\neq 0$ is not enough to guarantee $df(a)$ exists.
\end{remark}

\begin{theorem}
If $f$ is differentiable at a then $f$ is continuious at a.
\begin{proof}
\begin{align*}
\lim_{h \to 0}|f(a+h)-f(a)| &= \lim_{h \to 0}|f(a+h)-f(a) -Df(a) + Df(a)|\\
&\leq \lim_{h \to 0}\frac{|f(a+h)-f(a) -Df(a)(h)|}{|h|}\cdot|h| + \lim_{h \to 0}| Df(a)(h)|\\ 
&= 0 \\
\end{align*}
\end{proof}
\end{theorem}

\subsection{The Chain Rule}
\begin{theorem}[Chain Rule]\label{T:Chain Rule}
if $f:\RR^{n} \rightarrow \RR^{m}$ is differentiable at a and $f:\RR^{m} \rightarrow \RR^{k}$ is differentiable at $f(a)$ then $g \circ f:\RR^{n} \rightarrow \RR^{k}$ is differentiable at a and 
\[D(g \circ f)(a) = Dg(f(a))\circ Df(a)\]

\[\begin{diagram}
\node{\RR^{n}} \arrow{e,t}{Df(a)}  \arrow{se,b}{D(g\circ f)(a)}
\node{\RR^m}  \arrow{s,r}{Dg(f(a))} \\
 \node[2]{\RR^{k}}
\end{diagram}\]
\[(g\circ f)^{'}(a)= g^{'}(f(a))\cdot f^{'}(a), \quad \textrm{where $\cdot$ represents matrix multiplication}\]

\begin{proof}
if $b=f(a)$ and we let $Df(a) = \lambda$ and $Dg(f(a))=\mu$ then if we define:
\begin{align}
\varphi(x) &=f(x)-f(a) -\lambda(x-a)\\
\psi(y) &= g(y) - g(b) - \mu(y-b)\\
\rho(x) &= g\circ f(x) - g\circ f(a) - \mu \circ \lambda (x -a)
\end{align}
\text{Then:}
\begin{align}
\lim_{h \to 0}\frac{|f(a+h)-f(a) -Df(a)(h)|}{|h|} &= \lim_{x \to a}\frac{|\varphi(x)|}{|x-a|} = 0\\
\lim_{h \to 0}\frac{|g(b+h)-g(b) -Dg(b)(h)|}{|h|} &= \lim_{y \to b}\frac{|\psi(y)|}{|y-b|} = 0
\end{align}
We must show:
\[\lim_{h \to 0}\frac{|g\circ f(x) - g\circ f(a) - \mu \circ \lambda (x -a)|}{|h|} = \lim_{x \to b}\frac{|\rho(x)|}{|x-b|} = 0\]
Now:
\begin{align*}
\rho(x) &=g(f(x)) - g(b) - \mu(\lambda(x-a)) \\
&= g(f(x)) - g(b) - \mu(f(x) - f(a) - \varphi(x)) \qquad \; \textmd{by (1)}\\
&= [g(f(x)) - g(b) - \mu(\lambda(f(x)-f(a)))] \\
&= \mu(\varphi(x)) = \psi(f(x)) + \mu(f(x)) \qquad \qquad \qquad \quad \textmd{by (2)}
\end{align*}
Thus we must Prove
\begin{align}
&\lim_{x \to a}\frac{|\psi(f(x))|}{|x-a|} = 0\\
&\lim_{x \to a}\frac{|\mu\varphi(x)|}{|x-a|} = 0
\end{align}
It follows from (5) that for some $\delta > 0 $  we have
\[|\psi(f(x))|<\epsilon|f(x) - b| \quad if\; |f(x) - b|<\delta\]
which is true if $|x-a|< \delta_{1} $ for a suitable $\delta_{1}$. We also have that if T is a linear transformation then $\exists M \geq 0\; such \; that\;  |T(x)|<M|x|$. So then:
\begin{align*}
|\psi(f(x))| &<\epsilon|f(x) - b| \\
&= \epsilon|\varphi(x) + \lambda(x-a)|\\
& \leq \epsilon|\varphi(x)| + \epsilon M|x-a|
\end{align*}
So
\[\lim_{x \to a}\frac{|\psi(f(x))|}{|x-a|} \leq \lim_{x \to a}\frac{\epsilon|\varphi(x)|}{|x-a|}  + \lim_{x \to a}\frac{\epsilon M|x-a|}{|x-a|} = \epsilon M \rightarrow 0\]
Also
\[\lim_{x \to a}\frac{|\mu\varphi(x)|}{|x-a|} \leq \lim_{x \to a}\frac{M|\varphi(x)|}{|x-a|} = 0\]
\end{proof}
\end{theorem}

\begin{theorem}\label{s}
Define $s:\RR^2 \rightarrow \RR \quad s(x,y)=x + y$ then $s$ is differentiable  and $Ds = s$
\begin{proof}
S is linear so
\begin{align*}
s((x,y) + (x^{'},y^{'})) &=s(x+x^{'},y+y^{'})= s(x,y) + s(x^{'},y^{'})\\
s(\lambda(x,y)) &= \lambda s(x,y)\\
\lim_{h \to 0}\frac{|s(a+h) - s(a) -s(h)|}{|h|} = 0
\end{align*}
\end{proof}
\end{theorem}

\begin{theorem}\label{p}
Define $p:\RR^2 \rightarrow \RR, \quad p(x,y)=xy$, then $p$ is differentiable  and:\\
$Dp(a,b):\RR^2 \rightarrow \RR$ is linear with $Dp(a,b)(h,k) = ak + bh$ and $p^{'} = (b,a)$
\begin{proof}
use of derivitive
\begin{align*}
p((a,b)+(h,k)) - p(a,b) - Dp(a,b)(h,k) &= p(a+h,b+k) - p(a,b) - (ak + bh)\\
&=(a+h)(b+k) - ab - (ak + bh) = hk\\
\frac{|p((a,b)+(h,k)) - p(a,b) - Dp(a,b)(h,k)|}{|(h,k)|} &= \frac{|hk|}{\sqrt{h^2 + k^2}} \leq  \frac{\sqrt{h^2 + k^2}\sqrt{h^2 + k^2}}{\sqrt{h^2 + k^2}} = \sqrt{h^2 + k^2} \rightarrow 0
\end{align*}
\end{proof}
\end{theorem}

\begin{remark}
To check some $T:\RR^n \rightarrow \RR^m$ is linear we listed two properties:
\begin{align*}
T(x+y) &= T(x) +T(y)\\
T(\lambda x) &= \lambda T(x)\\
\text{we can instead just check:}\\
T( \lambda x + y) &= \lambda T(x) + T(y)
\end{align*}
\end{remark}

\subsection{Linear Functionals}

\begin{definition}
Let $g^{i}:\RR^n \rightarrow \RR$ be a linear map, such a map is called a linear functional. The set of all linear functionals from $\RR^n \rightarrow \RR$ is called the dual space of $\RR^{n}$, denoted $(\RR^n)*$\\
let $g^{1}, \dots , g^{m}$ be linear functionals $g^{i}:\RR^n \rightarrow \RR$, then I can combine them to get a map $g:\RR^n \rightarrow \RR^{m}$ by $g(x) =(g^{1}(x), \dots , g^{m}(x)$)\\
$g:\RR^n \rightarrow \RR^{m}$ is linear such for $x,y \in \RR^n, \; \lambda \in \RR$
\begin{align*}
g(\lambda x +y) &= \lambda g(x) + g(y) \\
\text{this can be seen by} \qquad
g(\lambda x +y) &=(g^{1}( \lambda x + y), \dots , g^{m}( \lambda x + y) \\
&= ( \lambda g(x)^{1} +g^{1}(y), \dots ,\lambda g(x)^{m} +g^{m}(y))\\
&= \lambda (g^{1}(x), \dots, g^{m}(x)) + (g^{1}, \dots , g^{m})
\end{align*}
$[g^{i}]$ is the matrix representation of $g^{i}$\\
$[g^{i}] = (g_{1}^{i}, \dots , g_{n}^{i})$
\[ [g]_{mxn} = \begin{pmatrix}
  g_{1}^{1}  & \cdots & g_{n}^{1} \\
  \vdots   & & \vdots  \\
  g_{1}^{m} & \cdots &g_{n}^{m}
 \end{pmatrix}\]
\end{definition}

\begin{theorem}
$f:\RR^{n} \rightarrow \RR^{m}$ is differentiable at a iff $f^{i}$ are differentiable at a, $i=1, \dots , m$
and $Df(a) = (Df^{1}, \dots , Df^{m}(a))$
\begin{proof}
assume f is differentiable at $a$ we take the linear function $\Pi^{i}(x^{1}, \dots , x^{m}) = x^{i}$ and compose it with $f$ we get 
\[f^{i} =  \Pi^{i} \circ f\]
this is differentiable by chain rule since $f$ and $\Pi^{i}$ are differentiable $\forall i =1 , \dots , m$
\[ \implies Df^{i} = D\Pi^{i}(a) \cdot Df(a)\]
$D\Pi^{i} = \Pi^{i}$
\[ \implies Df^{i} = \Pi^{i}(a) \cdot Df(a) \]
Now assume the all $f^{i}$ are differentiable at $a$ $\forall i=1, \dots , m$
\begin{align*}
&f(a+h) - f(a) -(Df^{1}(a)(h), \dots , Df^{m}(a)(h))\\
 &= (f^{1}(a+h), \dots , f^{m}(a+h)) - (f^{1}, \dots, f^{m}) - (Df^{1}*(a)(h), \dots , Df^{m}(a)(h))\\
&=(f^{1}(a+h) -f^{1}(a) - df^{1}(a) , \dots , f^{m}(a+h) -f^{m}(a) - df^{m}(a))
\end{align*}
So
\begin{align*}
 &\frac{|f(a+h) - f(a) -(Df^{1}(a)(h), \dots , Df^{m}(a)(h))|}{|h|} \\
&\leq \frac{|f^{1}(a+h) -f^{1}(a) - df^{1}(a)|}{|h|} , \dots ,\frac{| f^{m}(a+h) -f^{m}(a) - df^{m}(a)|}{|h|} \rightarrow 0
\end{align*}
\end{proof}
\end{theorem} 

\begin{remark}
If $T,S:\RR^{n} \rightarrow \RR^{m}$ are linear then $(T +S):\RR^{n} \rightarrow \RR^{m}, (T+S)(x) = T(x) + S(x)$ is linear.\\
If $\lambda \in \RR$ then $(\lambda T):\RR^{n} \rightarrow \RR^{m}, (\lambda T)(x) = \lambda \cdot T(x)$ is also linear.
\end{remark}

\begin{corollary}
$f,g:\RR^{n} \rightarrow \RR$ differentiable at $a \in \RR^{n}$
\begin{enumerate}[(i)]
\item $D(f+g)(a) = Df(a) + Dg(a)$
\item Product rule: $D(f \cdot g)(a) = g(a)\cdot Df(a) + f(a) \cdot Dg(a)$
\item Quotient rule: if $g(a) \neq 0, \; D(\frac{f}{g})(a) = \frac{1}{g(a)^2}\cdot (g(a)\cdot Df(a) - f(a) \cdot Dg(a))$
\end{enumerate}
\end{corollary}
\begin{proof}
For (i):\\
We can consder the function $s$ from theorem~\ref{s}, $s:\RR^2 \rightarrow \RR \quad s(x,y)=x + y$, but acting on $f$ and $g$ ie $s(f,g) = f+g$ and $Ds =s $
\[D(f+g)(a)= Ds(f(a),g(a)) \circ D(f,g)(a) = s \circ(Df(a),Dg(a)) = Df(a) + Dg(a)\]
For (ii):\\
We can consder the function $p$ from theorem~\ref{p}, $p:\RR^2 \rightarrow \RR \quad p(x,y)=xy$, but acting on $f$ and $g$ ie $p(f,g) = fg$ with  $Dp(f,g)(h,k) = fk + gh$
\[D(f \cdot g)(a) = Dp(f,g)\cdot D(f,g)(a)= Dp(f(a),g(a)) \cdot (Df(a),Dg(a)) = f(a)\cdot Dg(a) + g(a)\cdot Df(a)\]
(iii) follows from (ii)
\end{proof}



\end{document}
