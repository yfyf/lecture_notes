\documentclass[12pt]{article}
\usepackage{amsfonts, amsthm, amsmath}
\usepackage{verbatim} 
\usepackage{enumerate}
\setlength{\textwidth}{6.5in}
\setlength{\oddsidemargin}{0in}
\setlength{\textheight}{9.5in}
\setlength{\topmargin}{0in}
\setlength{\headheight}{0in}
\setlength{\headsep}{0in}
\setlength{\parskip}{0pt}
\setlength{\parindent}{0pt}

\def\CC{\mathbb{C}}
\def\FF{\mathbb{F}}
\def\PP{\mathbb{P}}
\def\QQ{\mathbb{Q}}
\def\RR{\mathbb{R}}
\def\ZZ{\mathbb{Z}}
\def\gotha{\mathfrak{a}}
\def\gothb{\mathfrak{b}}
\def\gothm{\mathfrak{m}}
\def\gotho{\mathfrak{o}}
\def\gothp{\mathfrak{p}}
\def\gothq{\mathfrak{q}}
\DeclareMathOperator{\disc}{Disc}
\DeclareMathOperator{\Gal}{Gal}
\DeclareMathOperator{\GL}{GL}
\DeclareMathOperator{\Hom}{Hom}
\DeclareMathOperator{\Norm}{Norm}
\DeclareMathOperator{\Trace}{Trace}
\DeclareMathOperator{\Cl}{Cl}

\def\head#1{\medskip \noindent \textbf{#1}.}

\newtheorem{theorem}{Theorem}
\newtheorem{lemma}[theorem]{Lemma}
\newtheorem{mydef}{Definition}
\newtheorem{example}{Example}
\newtheorem{prop}{Proposition}
\begin{document}

\title{Measure Theory}
\author{Prof. D Larman}
\date{Autum 2011}
\maketitle

\head{Introduction}  In this course we first seek to define the measure of a set eg. the length, area, volume, probability of a set. We also seek to improve on the riemann integral by defining the lebesgue integral.\\
 If $\lambda$ denotes the "length" of a set in  $\RR$., clearly we would expect $\lambda$[0,1]=1. But what about the length of [0,1]$\backslash\QQ$ where $\QQ$ is the set of rationals? Or the set $\bigcup_{i=0}^{\infty}[\frac{1}{2^{i+1}} +\frac{1}{2^i} ]$? Since $\QQ$ is quite "small" we might expect $\lambda$([0,1]$\backslash\QQ$)=1. Also we might expect\\  $\lambda$($\bigcup_{i=0}^{\infty}[\frac{1}{2^{i+1}} +\frac{1}{2^i} ]$)=$\sum_{i=0}^{\infty}\lambda$([$\frac{1}{2^{i+1}} +\frac{1}{2^i} ]$.  Both expectations are true!\\\\
If we take the function  $f(x) = \left\{ \begin{array}{rcl}
1 & \mbox{for} &
 \mbox{\emph{x  irrational}} \\ 0 & \mbox{for} & \mbox{\emph{x rational}} \\
\end{array}\right.$ \\then you will know from analysis 2 that ($\mathbf{R}$)$\int\limits_{0}^{-1}f(x)\,dx =1$ and ($\mathbf{R}$)$\int\limits_{-0}^{1}f(x)\,dx =1$
\\ however the vast majority of x in [0,1] are irrational and and so we might expect the integral to be 1. When we have defined the labesgue integral we will find ($\mathbf{L}$)$\int_{0}^{1}f(x)\,dx =1$

\begin{center}
\section{Measures}
\end{center}

We will work within a set $\Omega$. For example  $\Omega$ = $\RR$, $\Omega$ = $\RR^n$, $\Omega$ =$ \{sequence of heads \& tails\}$. Families of subsets of $\Omega$ will be denoted by $\mathcal{F}$,  $\mathcal{G}$ etc.\\

\begin{mydef}{Algebra of sets:} \\ A family $\mathcal{F}$ of subsets of $\Omega$ is called an Algebra if it satisfies:
\begin{enumerate}[(i)]
  \item $\phi, \Omega \in \mathcal{F}$
  \item If A $\in \mathcal{F}$ then $A^{c}= \Omega \backslash A \in \mathcal{F}$
  \item If A,B $\in \mathcal{F}$ then A$\cup$B $\in \mathcal{F}$
\end{enumerate}
 \end{mydef}

\begin{example} If $\Omega$=[0,1] and $\mathcal{F}$ is the family of all subsets of [0,1] which can be expressed as a finite union of intervals (which can be open, closed half open, empty) then $\mathcal{F}$ is an algebra. \end{example}

\begin{mydef}{$\sigma$-Algebra of sets:} \\ A family $\mathcal{F}$ of subsets of $\Omega$ is called a $\sigma$-Algebra if it satisfies:
\begin{enumerate}[(i)]
  \item $\phi, \Omega \in \mathcal{F}$
  \item If A $\in \mathcal{F}$ then $A^{c}= \Omega \backslash A \in \mathcal{F}$
  \item If $A_1,A_2,\dots$ is a sequence of sets in $\mathcal{F}$ then $\bigcup_{i=1}^{\infty} A_{i} \in \mathcal{F}$
\end{enumerate}
 \end{mydef}

\begin{example} 
For any $\Omega$.
\item $\mathcal{F}$=\{$\phi,\Omega$\} is a $\sigma$-algebra.
\item $\mathcal{F}$=\{all subsets of $\Omega$\} is a $\sigma$-algebra.
\end{example}

{\bf Remark:} althouth example 1 is an algebra, it is not a $\sigma$-algebra (try to prove it). Notice that a $\sigma$-algebra is an algebra. 
\vspace{20pt}
\begin{theorem}{De Morgan's Laws} If $A_{\alpha}$, $\alpha \in I$ is a family of sets in$\Omega$ then
\item ($\cup_{\alpha \in I}A_{\alpha})^{c}$=$\cap_{\alpha \in I}A_{\alpha}^{c}$
\item ($\cap_{\alpha \in I}A_{\alpha})^{c}$=$\cup_{\alpha \in I}A_{\alpha}^{c}$
\end{theorem}


From the defintion of an algebra or a $\sigma$-algebra we can deduce the following properties:
\vspace{20pt}\\
{\bf Algebra}
\begin{enumerate}[(i)]
  \item $A_{i}, i=1,2,\dots,n \in \mathcal{F} \implies \bigcup_{i=1}^{n} A_{i} \in \mathcal{F}$ (induction)
  \item  $A_{i}, i=1,2,\dots,n \in \mathcal{F} \implies \bigcap_{i=1}^{n} A_{i} \in \mathcal{F}$ (By De Morgan (ii))
  \item A,B $\in \mathcal{F} \implies A\backslash B\in \mathcal{F}$ (Since $A\backslash B=A\cap B^{c}$)
\end{enumerate}

\vspace{12pt}
{\bf $\sigma$-Algebra}
\begin{enumerate}[(i)]
  \item  $A_{1}, A_{2},\dots \in \mathcal{F}$ then $\bigcap_{i=1}^{\infty} A_{i} \in \mathcal{F}$  ($\bigcap_{i=1}^{\infty} A_{i} = \bigcap_{i=1}^{\infty}( A_{i}^{c})^{c}= (\bigcup_{i=1}^{\infty} A_{i}^{c})^{c}\in \mathcal{F})$
\end{enumerate}
\vspace{12pt}
\begin{prop}($\sigma$-algebra generated by A)\\ For any family of subsets A of $\Omega$, there is a smallest $\sigma$-algebra $\sigma$(A) containing A.\\ i.e. $\sigma(A) \supset A$ and if $\mathcal{F}$ is any $\sigma$-algebra containing A then 
$\sigma(A)\subset\mathcal{F}$. We call $\sigma(A)$ the $\sigma$-algebra generated by A.
\end{prop}

\begin{proof}
Just note that there is a $\sigma$-algebra containing A, namely \{all subsets of A\}.\\ Consider all $\sigma$-algebras containing A and let $\sigma(A)$ be their intersection. i.e. B$\in\sigma(A)$ iff B belongs to every $\sigma$-algebra containing A. We certainly have A$\subset\sigma(A)$ and if $\mathcal{F}$ is a $\sigma$-algebra containing A then $\sigma(A)\subset\mathcal{F}$. It remains to show that $\sigma(A)$ is a $\sigma$-algebra.
\begin{enumerate}[(i)]
  \item $\phi, \Omega \in \sigma(A)$ since they belong to every $\sigma$-algebra containing A.
  \item  If A$\in \sigma(A)$ and $\mathcal{F}$ is a $\sigma$-algebra containing A, then A$\in \mathcal{F}$ and so $A^{c}\in\mathcal{F}$.\\ So $A^{c}\in\sigma(A)$
  \item If \{$A_{i}\}_{i=1}^{\infty}\in\sigma(A)$ and $\mathcal{F}$ is a $\sigma$-algebra containing A then  \{$A_{i}\}_{i=1}^{\infty}\in\mathcal{F}$ \& so $\bigcup_{i=1}^{\infty} A_{i}\in\mathcal{F}$. Hence $\bigcup_{i=1}^{\infty} A_{i}\in\sigma(A)$.
\end{enumerate}
\end{proof}

The most important $\sigma$-algebra is the:

\begin{mydef}{Borel $\sigma$-algebra:}\\
This is the $\sigma$-algebra on $\RR$ generated by the family of open intervals in $\RR$.
\end{mydef}

\begin{mydef}{Borel Set:}\\
A Borel Set is any set which belongs to the Borel $\sigma$-algebra eg.\\ $\phi, \RR$, any open interval, any closed interval $([a,b]=\bigcap_{i=1}^{\infty}(a - \frac{1}{i}, b + \frac{1}{i})).$\\ Most reasonable sets are Borel:\\
$[a,b)=\bigcap_{n=1}^{\infty}(a - \frac{1}{n}, b)$, $\{a\}=\bigcap_{n=1}^{\infty}(a - \frac{1}{n}, a + \frac{1}{n})$, $\QQ=\bigcup_{n=1}^{\infty}r_n$, $I(irrationals)=\QQ^{c}$.
\end{mydef}
\end{document}
