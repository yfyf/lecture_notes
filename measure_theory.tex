\documentclass[12pt]{article}
\usepackage{amsfonts, amsthm, amsmath}
\usepackage{verbatim} 
\setlength{\textwidth}{6.5in}
\setlength{\oddsidemargin}{0in}
\setlength{\textheight}{9.5in}
\setlength{\topmargin}{0in}
\setlength{\headheight}{0in}
\setlength{\headsep}{0in}
\setlength{\parskip}{0pt}
\setlength{\parindent}{0pt}

\def\CC{\mathbb{C}}
\def\FF{\mathbb{F}}
\def\PP{\mathbb{P}}
\def\QQ{\mathbb{Q}}
\def\RR{\mathbb{R}}
\def\ZZ{\mathbb{Z}}
\def\gotha{\mathfrak{a}}
\def\gothb{\mathfrak{b}}
\def\gothm{\mathfrak{m}}
\def\gotho{\mathfrak{o}}
\def\gothp{\mathfrak{p}}
\def\gothq{\mathfrak{q}}
\DeclareMathOperator{\disc}{Disc}
\DeclareMathOperator{\Gal}{Gal}
\DeclareMathOperator{\GL}{GL}
\DeclareMathOperator{\Hom}{Hom}
\DeclareMathOperator{\Norm}{Norm}
\DeclareMathOperator{\Trace}{Trace}
\DeclareMathOperator{\Cl}{Cl}

\def\head#1{\medskip \noindent \textbf{#1}.}

\newtheorem{theorem}{Theorem}
\newtheorem{lemma}[theorem]{Lemma}
\newtheorem{mydef}{Definition}
\newtheorem{example}{Example}
\newtheorem{prop}{Proposition}
\begin{document}

\begin{titlepage}
\begin{center}
\textsc{\LARGE Measure Theory }\\[1.5cm]

\textsc{\Large Prof. D Larman}\\[0.5cm]
\end{center}
\end{titlepage}

\head{Introduction}  In this course we first seek to define the measure of a set eg. the length, area, volume, probability of a set. We also seek to improve on the riemann integral by defining the lebesgue integral.\\
 If $\lambda$ denotes the "length" of a set in  $\RR$., clearly we would expect $\lambda$[0,1]=1. But what about the length of [0,1]$\backslash\QQ$ where $\QQ$ is the set of rationals? Or the set $\bigcup_{i=0}^{\infty}[\frac{1}{2^{i+1}} +\frac{1}{2^i} ]$? Since $\QQ$ is quite "small" we might expect $\lambda$([0,1]$\backslash\QQ$)=1. Also we might expect\\  $\lambda$($\bigcup_{i=0}^{\infty}[\frac{1}{2^{i+1}} +\frac{1}{2^i} ]$)=$\sum_{i=0}^{\infty}\lambda$([$\frac{1}{2^{i+1}} +\frac{1}{2^i} ]$.  Both expectations are true!\\\\
If we take the function  $f(x) = \left\{ \begin{array}{rcl}
1 & \mbox{for} &
 \mbox{\emph{x  irrational}} \\ 0 & \mbox{for} & \mbox{\emph{x rational}} \\
\end{array}\right.$ \\then you will know from analysis 2 that ($\mathbf{R}$)$\int\limits_{0}^{-1}f(x)\,dx =1$ and ($\mathbf{R}$)$\int\limits_{-0}^{1}f(x)\,dx =1$
\\ however the vast majority of x in [0,1] are irrational and and so we might expect the integral to be 1. When we have defined the labesgue integral we will find ($\mathbf{L}$)$\int_{0}^{1}f(x)\,dx =1$

\begin{center}
\section{Measures}
\end{center}

We will work within a set $\Omega$. For example  $\Omega$ = $\RR$, $\Omega$ = $\RR^n$, $\Omega$ =$ \{sequence of heads \& tails\}$. Families of subsets of $\Omega$ will be denoted by $\mathcal{F}$,  $\mathcal{G}$ etc.\\

\begin{mydef}{Algebra of sets:} \\ A family $\mathcal{F}$ of subsets of $\Omega$ is called an Algebra if it satisfies:
\begin{itemize}
  \item[(i)] $\phi, \Omega \in \mathcal{F}$
  \item[(ii)] If A $\in \mathcal{F}$ then $A^{c}= \Omega \backslash A \in \mathcal{F}$
  \item[(iii)] If A,B $\in \mathcal{F}$ then A$\cup$B $\in \mathcal{F}$
\end{itemize}
 \end{mydef}

\begin{example} If $\Omega$=[0,1] and $\mathcal{F}$ is the family of all subsets of [0,1] which can be expressed as a finite union of intervals (which can be open, closed half open, empty) then $\mathcal{F}$ is an algebra. \end{example}

\begin{mydef}{$\sigma$-Algebra of sets:} \\ A family $\mathcal{F}$ of subsets of $\Omega$ is called a $\sigma$-Algebra if it satisfies:
\begin{itemize}
  \item[(i)] $\phi, \Omega \in \mathcal{F}$
  \item[(ii)] If A $\in \mathcal{F}$ then $A^{c}= \Omega \backslash A \in \mathcal{F}$
  \item[(iii)] If $A_1,A_2,\dots$ is a sequence of sets in $\mathcal{F}$ then $\bigcup_{i=1}^{\infty} A_{i} \in \mathcal{F}$
\end{itemize}
 \end{mydef}

\begin{example} 
For any $\Omega$.
\item $\mathcal{F}$=\{$\phi,\Omega$\} is a $\sigma$-algebra.
\item $\mathcal{F}$=\{all subsets of $\Omega$\} is a $\sigma$-algebra.
\end{example}

{\bf Remark:} althouth example 1 is an algebra, it is not a $\sigma$-algebra (try to prove it). Notice that a $\sigma$-algebra is an algebra. 
\vspace{20pt}
\begin{theorem}{De Morgan's Laws} If $A_{\alpha}$, $\alpha \in I$ is a family of sets in$\Omega$ then
\item ($\cup_{\alpha \in I}A_{\alpha})^{c}$=$\cap_{\alpha \in I}A_{\alpha}^{c}$
\item ($\cap_{\alpha \in I}A_{\alpha})^{c}$=$\cup_{\alpha \in I}A_{\alpha}^{c}$
\end{theorem}


From the definition of an algebra or a $\sigma$-algebra we can deduce the following properties:
\vspace{20pt}\\
{\bf Algebra}
\begin{itemize}
  \item[(i)] $A_{i}, i=1,2,\dots,n \in \mathcal{F} \implies \bigcup_{i=1}^{n} A_{i} \in \mathcal{F}$ (induction)
  \item[(ii)]  $A_{i}, i=1,2,\dots,n \in \mathcal{F} \implies \bigcap_{i=1}^{n} A_{i} \in \mathcal{F}$ (By De Morgan (ii))
  \item[(iii)] A,B $\in \mathcal{F} \implies A\backslash B\in \mathcal{F}$ (Since $A\backslash B=A\cap B^{c}$)
\end{itemize}

\vspace{12pt}
{\bf $\sigma$-Algebra}
\begin{itemize}
  \item[(i)]  $A_{1}, A_{2},\dots \in \mathcal{F}$ then $\bigcap_{i=1}^{\infty} A_{i} \in \mathcal{F}$  ($\bigcap_{i=1}^{\infty} A_{i} = \bigcap_{i=1}^{\infty}( A_{i}^{c})^{c}= (\bigcup_{i=1}^{\infty} A_{i}^{c})^{c}\in \mathcal{F})$
\end{itemize}
\vspace{12pt}
\begin{prop}($\sigma$-algebra generated by A)\\ For any family of subsets A of $\Omega$, there is a smallest $\sigma$-algebra $\sigma$(A) containing A.\\ i.e. $\sigma(A) \supset A$ and if $\mathcal{F}$ is any $\sigma$-algebra containing A then 
$\sigma(A)\subset\mathcal{F}$. We call $\sigma(A)$ the $\sigma$-algebra generated by A.
\end{prop}

\begin{proof}
Just note that there is a $\sigma$-algebra containing A, namely \{all subsets of A\}.\\ Consider all $\sigma$-algebras containing A and let $\sigma(A)$ be their intersection. i.e. B$\in\sigma(A)$ iff B belongs to every $\sigma$-algebra containing A. We certainly have A$\subset\sigma(A)$ and if $\mathcal{F}$ is a $\sigma$-algebra containing A then $\sigma(A)\subset\mathcal{F}$. It remains to show that $\sigma(A)$ is a $\sigma$-algebra.
\begin{itemize}
  \item[(i)] $\phi, \Omega \in \sigma(A)$ since they belong to every $\sigma$-algebra containing A.
  \item[(ii)]  If A$\in \sigma(A)$ and $\mathcal{F}$ is a $\sigma$-algebra containing A, then A$\in \mathcal{F}$ and so $A^{c}\in\mathcal{F}$.\\ So $A^{c}\in\sigma(A)$
  \item[(iii)] If \{$A_{i}\}_{i=1}^{\infty}\in\sigma(A)$ and $\mathcal{F}$ is a $\sigma$-algebra containing A then  \{$A_{i}\}_{i=1}^{\infty}\in\mathcal{F}$ \& so $\bigcup_{i=1}^{\infty} A_{i}\in\mathcal{F}$. Hence $\bigcup_{i=1}^{\infty} A_{i}\in\sigma(A)$.
\end{itemize}
\end{proof}

The most important $\sigma$-algebra is the:

\begin{mydef}{Borel $\sigma$-algebra:}\\
This is the $\sigma$-algebra on $\RR$ generated by the family of open intervals in $\RR$.
\end{mydef}

\begin{mydef}{Borel Set:}\\
A Borel Set is any set which belongs to the Borel $\sigma$-algebra eg.\\ $\phi, \RR$, any open interval, any closed interval $([a,b]=\bigcap_{i=1}^{\infty}(a - \frac{1}{i}, b + \frac{1}{i})).$\\ Most reasonable sets are Borel:\\
$[a,b)=\bigcap_{n=1}^{\infty}(a - \frac{1}{n}, b)$, $\{a\}=\bigcap_{n=1}^{\infty}(a - \frac{1}{n}, a + \frac{1}{n})$, $\QQ=\bigcup_{n=1}^{\infty}r_n$, $I(irrationals)=\QQ^{c}$.
\end{mydef}































\begin{comment}
\head{Jargon watch} If $G$ is a group, a \emph{$G$-extension} of a field
$K$ is a Galois extension of $K$ with Galois group $G$.

Before attempting to classify all abelian extensions of $\QQ_p$,
we recall an older classification result due to Kummer. This result will
continue to be useful as we proceed to class field theory in general, and 
the technique in its proof prefigures the role to be played
by group cohomology down the line. So watch carefully!

\begin{theorem}[Kummer]
If $\zeta_n \in K$, then every $\ZZ/n\ZZ$-extension of $K$ is of the form
$K(\alpha^{1/n})$ for some $\alpha \in K^*$ with the property that
$\alpha^{1/d} \notin K$ for any proper divisor $d$ of $n$, and vice versa.
\end{theorem}

I don't actually know how Kummer proved this; nowadays one uses the following.
If $G$ is a group and $M$ is an abelian group on which $G$ acts
(written multiplicatively),
one defines the group $H^1(G,M)$ as the set of functions $f:
G \to M$ such that $f(gh) = f(g)^h f(h)$, modulo the set of such
functions of the form $f(g) = x^g x^{-1}$ for some $x \in M$.
(This is our first example of \emph{group cohomology}; more on this 
later. Also, beware that ``Hilbert's Theorem 90'' was only proved by
Hilbert when $G$ is cyclic; the general case is due to Emmy Noether.)
\begin{lemma}[``Hilbert's Theorem 90'']
Let $L/K$ be a finite Galois extension with Galois group $G$.
Then $H^1(G, L^*) = 0$.
\end{lemma}
\begin{proof}
Let $f$ be a function of the form described above.
By the linear independence of automorphisms (see exercises),
there exists $x \in L$ such that $t = \sum_{g \in G} x^g f(g)$
is nonzero. But now
\[
t^h = \sum_{g \in G} x^{gh} f(g)^h =
\sum_{g \in G} x^{gh} f(gh) f(h)^{-1}
= f(h)^{-1} t.
\]
Thus $f$ is zero in $H^1(G,L^*)$.
\end{proof}

\begin{proof}[Proof of Kummer's Theorem]
On one hand, if $\alpha \in K^*$ is such that $\alpha^{1/d} \notin K$
for any proper divisor $d$ of $n$, then the polynomial $x^n - \alpha$
is irreducible over $K$, and every automorphism must have the form
$\alpha \mapsto \alpha \zeta_n^r$ for some $r \in \ZZ/n\ZZ$. Thus
$\Gal(K(\alpha^{1/n})/K) \cong \ZZ/n\ZZ$.

On the other hand,
let $L$ be an arbitrary
$\ZZ/n\ZZ$-extension of $K$. Choose a generator $g \in \Gal(L/K)$,
and let $f: \Gal(L/K) \to L^*$ be the map that sends $rg$ to $\zeta_n^r$
for $r \in \ZZ$.
Then $f \in H^1(\Gal(L/K), L^*)$, so there exists $t \in L$ such that
$t^{rg}/t = f(rg) = \zeta_n^r$ for $r \in \ZZ$. In particular,
$t^n$ is invariant under $\Gal(L/K)$, so $t^n = \alpha$ for some
$\alpha \in K$ and $L = K(t) = K(\alpha^{1/n})$, as desired.
\end{proof}

Another way to state Kummer's theorem is as a bijection
\[
\mbox{$(\ZZ/n\ZZ)^r$-extensions of $K$} \leftrightarrow
\mbox{$(\ZZ/n\ZZ)^r$-subgroups of $K^*/(K^*)^n$}.
\]
(What we proved above was the case $r=1$, but the general case follows easily.)
Another way is in terms of the absolute Galois group of $K$.
Define the \emph{Kummer pairing}
\[
\langle \cdot, \cdot \rangle:
\Gal(\overline{K}/K) \times K^* \to \{1, \zeta_n, \dots, \zeta_n^{n-1} \}
\]
as follows: given $\sigma \in \Gal(\overline{K}/K)$
and $z \in K^*$, choose $y \in \overline{K}^*$ such that $y^n = z$,
and put $\langle \sigma, z \rangle = y^\sigma/y$. Note that this does not
depend on the choice of $y$: the other possibilities are $y \zeta_n^k$
for $k=0, \dots, n-1$, and $\zeta_n^\sigma = \zeta_n$ by the assumption
on $K$, so it drops out.

\begin{theorem}[Kummer reformulated]
The Kummer pairing induces an isomorphism
\[
K^*/(K^*)^n \to \Hom(\Gal(\overline{K}/K), \ZZ/n\ZZ).
\]
\end{theorem}
\begin{proof}
The map comes from the pairing; we have to check it's injective and
surjective. If $y \in K^* \setminus (K^*)^n$, then $K(y^{1/n})$ is a nontrivial
Galois extension of $K$, so there exists some element of
$\Gal(K(y^{1/n})/K)$ that doesn't preserve $y^{1/n}$. Any lift of
that element to $\Gal(\overline{K}/K)$ pairs with $y$ to give something
other than 1; that is, $y$ induces a nonzero homomorphism of
$\Gal(\overline{K}/K)$ to $\ZZ/n\ZZ$. Thus injectivity follows.

On the other hand, suppose $f: \Gal(\overline{K}/K) \to \ZZ/n\ZZ$ is a
homomorphism whose image is the cyclic subgroup of $\ZZ/n\ZZ$ of order $d$.
Let $H$ be the kernel of $f$; then the fixed field $L$ of $H$ is
a $\ZZ/d\ZZ$-extension of $K$ with Galois group $\Gal(\overline{K}/K)/H$.
By Kummer theory, $L = K(y^{1/d})$ for some $y$. But now the homomorphisms
induced by $y^{mn/d}$, as $m$ runs over all integers coprime to $d$,
give all possible homomorphisms of $\Gal(\overline{K}/K)/H$ to $\ZZ/d\ZZ$,
so one of them must equal $f$. Thus surjectivity follows.
\end{proof}

But what about $\ZZ/n\ZZ$-extensions of a field that does not contains
$\zeta_{n}$? These are harder to describe, and indeed describing such 
extensions of $\QQ$ is the heart of this course. There is one thing
one can say: if $L/K$ is a $\ZZ/n\ZZ$-extension, so is $L(\zeta_n)/K$, and
the latter is a Kummer extension.
\begin{lemma}
Let $n$ be a prime (or an odd prime power),
let $K$ be a field of characteristic coprime to $n$, let $L = K(\zeta_n)$,
and let $M = L(a^{1/n})$ for some $a \in L^*$. Define the homomorphism
$\omega: \Gal(L/K) \to (\ZZ/n\ZZ)^*$ by the relation
$\zeta_n^{\omega(g)} = \zeta_n^g$. Then $M/K$ is Galois and
abelian if and only if
\begin{equation} \label{eq}
a^g / a^{\omega(g)} \in (L^*)^n \qquad \forall g \in \Gal(M/K).
\end{equation}
\end{lemma}
Note that $\omega(g)$ is only defined up to adding a multiple of $n$,
so $a^{\omega(g)}$ is only defined up to an $n$-th power, i.e., modulo
$(M^*)^n$.
\begin{proof}
If $a^g/a^{\omega(g)} \in (L^*)^n$ for all $g \in \Gal(M/K)$,
then $a$, $a^{\omega(g)}$ and $a^g$ all generate the same subgroup
of $(L^*)/(L^*)^n$. Thus $L(a^{1/n}) = L((a^g)^{1/n})$ for all $g \in
\Gal(M/K)$, so $M/K$ is Galois. Thus it suffices to assume $M/K$ is
Galois, then prove that $M/K$ is abelian if and only if (\ref{eq}) holds.
In this case, we must have $a^g/a^{\rho(g)} \in (M^*)^n$ for some map
$\rho: \Gal(M/K) \to (\ZZ/n\ZZ)^*$, and the latter is cyclic by our
assumption on $n$.

Note that $\Gal(M/K)$ has a homomorphism $\omega$ to a cyclic group whose
kernel $\Gal(M/L) \subseteq \ZZ/n\ZZ$ is also abelian. Thus $\Gal(M/K)$
is abelian if and only if $g$ and $h$ commute for any $g \in \Gal(M/K)$
and $h \in \Gal(M/L)$, i.e., if $h = g^{-1}hg$.
(Since $g$ commutes with powers of itself, $g$ then
commutes with everything.)

Let $A \subseteq L^*/(L^*)^n$ be the subgroup generated by $a$. Then
the Kummer pairing gives rise to a pairing
\[
\Gal(M/L) \times A \to \{1, \zeta_{n-1}, \dots, \zeta_n^{n-1}\}
\]
which is bilinear and nondegenerate, so $h = g^{-1}hg$ if and only if
$\langle h, s^g \rangle = \langle ghg^{-1}, s^g \rangle$ for all
$s \in A$. But the Kummer pairing is \emph{equivariant} with respect
to $\Gal(L/K)$ as follows:
\[
\langle h,s \rangle^g = \langle g^{-1}hg, s^g \rangle,
\]
because
\[
\left( \frac{(s^{1/n})^h}{s^{1/n}} \right)^g
= \frac{((s^g)^{1/n})^{g^{-1}hg}}{(s^g)^{1/n}}.
\]
(Here by $s^{1/n}$ I mean an arbitrary $n$-th root of $s$ in $M$,
and by $(s^g)^{1/n}$ I mean $(s^{1/n})^g$. Remember the value of the
Kummer pairing doesn't depend on which $n$-th root you choose.)
Thus $h = ghg^{-1}$ if and only if $\langle h,s^g \rangle =
\langle h,s \rangle^g$ for all $s \in A$, or equivalently,
just for $s=a$.
But
\[
\langle h,a \rangle^g = \langle h,a \rangle^{\omega(g)}
= \langle h, a^{\omega(g)} \rangle.
\]
Thus $g$ and $h$ commute if and only if $\langle h, a^g \rangle
= \langle h, a^{\omega(g)}rangle$, if and only if (by nondegeneracy)
$a^g/a^{\omega(g)} \in (L^*)^n$, as desired.
\end{proof}

\head{Exercises}

\begin{enumerate}
\item
Prove the linear independence of automorphisms: if $g_1, \dots, g_n$
are distinct automorphisms of $L$ over $K$, then there do not exist
$x_1, \dots, x_n \in L$ such that $x_1 y^{g_1} + \cdots + x_n y^{g_n} = 0$
for all $y \in L$. (Hint: suppose the contrary, choose a counterexample
with $n$ as small as possible, then make an even smaller counterexample.)
\item
Prove the additive analogue of Hilbert's Theorem 90: if $L/K$ is a finite
Galois extension with Galois group $G$, then $H^1(G, L) = 0$, where the
abelian group is now the additive group of $L$.
(Hint: by the normal basis theorem
(see for example Lang, \emph{Algebra}),
there exists $\alpha \in L$ whose conjugates form a basis
of $L$ as a $K$-vector space.)
\item
(Optional) If $L/K$ is a finite Galois extension with Galois group $G$,
then $H^1(G, \GL(n,L)) = 0$. (Hint: the hard part is proving that the thing
that you get from independence of automorphisms is actually an 
\emph{invertible} matrix. Show in fact that the matrices you can get satisfy
no $L$-linear relations.)
\end{enumerate}
\end{comment}
\end{document}