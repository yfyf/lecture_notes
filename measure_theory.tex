\documentclass[12pt]{article}
\usepackage{amsfonts, amsthm, amsmath}
\usepackage{verbatim}
\usepackage{enumerate}
\usepackage{graphicx}
\usepackage{amssymb}
\usepackage{epstopdf}
\usepackage{amsthm}

\setlength{\textwidth}{6.5in}
\setlength{\oddsidemargin}{0in}
\setlength{\textheight}{10.5in}
\setlength{\topmargin}{0in}
\setlength{\headheight}{0in}
\setlength{\headsep}{0in}
\setlength{\parskip}{0pt}
\setlength{\parindent}{0pt}

\def\CC{\mathbb{C}}
\def\FF{\mathbb{F}}
\def\PP{\mathbb{P}}
\def\QQ{\mathbb{Q}}
\def\RR{\mathbb{R}}
\def\ZZ{\mathbb{Z}}
\def\gotha{\mathfrak{a}}
\def\gothb{\mathfrak{b}}
\def\gothm{\mathfrak{m}}
\def\gotho{\mathfrak{o}}
\def\gothp{\mathfrak{p}}
\def\gothq{\mathfrak{q}}
\DeclareMathOperator{\disc}{Disc}
\DeclareMathOperator{\Gal}{Gal}
\DeclareMathOperator{\GL}{GL}
\DeclareMathOperator{\Hom}{Hom}
\DeclareMathOperator{\Norm}{Norm}
\DeclareMathOperator{\Trace}{Trace}
\DeclareMathOperator{\Cl}{Cl}

\def\head#1{\medskip \noindent \textbf{#1}.}

\newtheorem{theorem}{Theorem}
\newtheorem{lemma}[theorem]{Lemma}
\newtheorem{definition}{Definition}
\newtheorem{example}{Example}
\newtheorem{proposition}{Proposition}
\newtheorem*{note}{Note}
\newtheorem*{claim}{Claim}
\newtheorem*{notation}{Notation}
\newtheorem{corollary}{Corollary}
\begin{document}
\begin{titlepage}
\title{Measure Theory}
\author{Prof. D Larman}
\date{Autum 2011}
\maketitle

\section*{Introduction} In this course we first seek to define the measure of a set eg. the length, area, volume, probability of a set. We also seek to improve on the riemann integral by defining the lebesgue integral.\\
 If $\lambda$ denotes the "length" of a set in $\RR$., clearly we would expect $\lambda$[0,1]=1. But what about the length of [0,1]$\backslash\QQ$ where $\QQ$ is the set of rationals? Or the set $\bigcup_{i=0}^{\infty}[\frac{1}{2^{i+1}} +\frac{1}{2^i} ]$? Since $\QQ$ is quite "small" we might expect $\lambda$([0,1]$\backslash\QQ$)=1. Also we might expect\\ \(\lambda(\bigcup_{i=0}^{\infty}[\frac{1}{2^{i+1}} +\frac{1}{2^i} ])=\sum_{i=0}^{\infty}\lambda([\frac{1}{2^{i+1}} +\frac{1}{2^i} ]\). Both expectations are true!\\\\
If we take the function $f(x) = \left\{ \begin{array}{rcl}
1 & \mbox{for} &
\mbox{\emph{x irrational}} \\ 0 & \mbox{for} & \mbox{\emph{x rational}} \\
\end{array}\right.$ \\then you will know from analysis 2 that ($\mathbf{R}$)$\int\limits_{0}^{-1}f(x)\,dx =1$ and ($\mathbf{R}$)$\int\limits_{-0}^{1}f(x)\,dx =1$
\\ however the vast majority of x in [0,1] are irrational and and so we might expect the integral to be 1. When we have defined the labesgue integral we will find ($\mathbf{L})\int_{0}^{1}f(x)\,dx =1$
\end{titlepage}
\begin{center}
\section{Measures}
\end{center}

We will work within a set $\Omega$. For example $\Omega$ = $\RR$, $\Omega$ = $\RR^n$, $\Omega$ = \{sequence of heads \& tails\}. Families of subsets of $\Omega$ will be denoted by $\mathcal{F}$, $\mathcal{G}$ etc.\\

\begin{definition}[Algebra of sets] A family $\mathcal{F}$ of subsets of $\Omega$ is called an Algebra if it satisfies:
\begin{enumerate}[(i)]
  \item $\phi, \Omega \in \mathcal{F}$
  \item If A $\in \mathcal{F}$ then $A^{c}= \Omega \backslash A \in \mathcal{F}$
  \item If A,B $\in \mathcal{F}$ then A$\cup$B $\in \mathcal{F}$
\end{enumerate}
 \end{definition}

\begin{example} If $\Omega$=[0,1] and $\mathcal{F}$ is the family of all subsets of [0,1] which can be expressed as a finite union of intervals (which can be open, closed half open, empty) then $\mathcal{F}$ is an algebra. \end{example}

\begin{definition}[$\sigma$-Algebra of sets] A family $\mathcal{F}$ of subsets of $\Omega$ is called a $\sigma$-Algebra if it satisfies:
\begin{enumerate}[(i)]
  \item $\phi, \Omega \in \mathcal{F}$
  \item If A $\in \mathcal{F}$ then $A^{c}= \Omega \backslash A \in \mathcal{F}$
  \item If $A_1,A_2,\dots$ is a sequence of sets in $\mathcal{F}$ then $\bigcup_{i=1}^{\infty} A_{i} \in \mathcal{F}$
\end{enumerate}
 \end{definition}

\begin{example}
For any $\Omega$.
\item $\mathcal{F}$=\{$\phi,\Omega$\} is a $\sigma$-algebra.
\item $\mathcal{F}$=\{all subsets of $\Omega$\} is a $\sigma$-algebra.
\end{example}

{\bf Remark:} althouth example 1 is an algebra, it is not a $\sigma$-algebra (try to prove it). Notice that a $\sigma$-algebra is an algebra.
\vspace{20pt}
\begin{theorem}[De Morgan's Laws] If $A_{\alpha}$, $\alpha \in I$ is a family of sets in$\Omega$ then
\begin{enumerate}[(i)]
\item \((\cup_{\alpha \in I}A_{\alpha})^{c}=\cap_{\alpha \in I}A_{\alpha}^{c}\)
\item \((\cap_{\alpha \in I}A_{\alpha})^{c}=\cup_{\alpha \in I}A_{\alpha}^{c}\)
\end{enumerate}
\end{theorem}


From the defintion of an algebra or a $\sigma$-algebra we can deduce the following properties:
\vspace{20pt}\\
{\bf Algebra}
\begin{enumerate}[(i)]
  \item \(A_{i}, i=1,2,\dots,n \in \mathcal{F} \implies \bigcup_{i=1}^{n} A_{i} \in \mathcal{F}$ (induction)
  \item \(A_{i}, i=1,2,\dots,n \in \mathcal{F} \implies \bigcap_{i=1}^{n} A_{i} \in \mathcal{F}$ (By De Morgan (ii))
  \item \(A,B \in \mathcal{F} \implies A\backslash B \in \mathcal{F}\) (Since \(A\backslash B=A\cap B^{c})\)
\end{enumerate}

\vspace{12pt}
{\bf $\sigma$-Algebra}
\begin{enumerate}[(i)]
  \item \(A_{1}, A_{2},\dots \in \mathcal{F}$ then $\bigcap_{i=1}^{\infty} A_{i} \in \mathcal{F} (\bigcap_{i=1}^{\infty} A_{i} = \bigcap_{i=1}^{\infty}( A_{i}^{c})^{c}= (\bigcup_{i=1}^{\infty} A_{i}^{c})^{c}\in \mathcal{F})\)
\end{enumerate}
\vspace{12pt}

\begin{proposition}\label{P:smallestsigmaalg}
  For any family of subsets A of $\Omega$, there is a smallest $\sigma$-algebra $\sigma$(A) containing A.\\ i.e. $\sigma(A) \supset A$ and if $\mathcal{F}$ is any $\sigma$-algebra containing A then
$\sigma(A)\subset\mathcal{F}$. We call $\sigma(A)$ the $\sigma$-algebra generated by A.
\end{proposition}

\begin{proof}
Just note that there is a $\sigma$-algebra containing A, namely \{all subsets of A\}.\\ Consider all $\sigma$-algebras containing A and let $\sigma(A)$ be their intersection. i.e. B$\in\sigma(A)$ iff B belongs to every $\sigma$-algebra containing A. We certainly have A$\subset\sigma(A)$ and if $\mathcal{F}$ is a $\sigma$-algebra containing A then $\sigma(A)\subset\mathcal{F}$. It remains to show that $\sigma(A)$ is a $\sigma$-algebra.
\begin{enumerate}[(i)]
  \item $\phi, \Omega \in \sigma(A)$ since they belong to every $\sigma$-algebra containing A.
  \item If A$\in \sigma(A)$ and $\mathcal{F}$ is a $\sigma$-algebra containing A, then A$\in \mathcal{F}$ and so $A^{c}\in\mathcal{F}$.\\ So $A^{c}\in\sigma(A)$
  \item If \{$A_{i}\}_{i=1}^{\infty}\in\sigma(A)$ and $\mathcal{F}$ is a $\sigma$-algebra containing A then \{$A_{i}\}_{i=1}^{\infty}\in\mathcal{F}$ \& so $\bigcup_{i=1}^{\infty} A_{i}\in\mathcal{F}$. Hence $\bigcup_{i=1}^{\infty} A_{i}\in\sigma(A)$.
\end{enumerate}
\end{proof}

The most important $\sigma$-algebra is the:

\begin{definition}[Borel $\sigma$-algebra]
This is the $\sigma$-algebra on $\RR$ generated by the family of open intervals in $\RR$.
\end{definition}

\begin{definition}[Borel Set]
A Borel Set is any set which belongs to the Borel $\sigma$-algebra eg.\\ $\phi, \RR$, any open interval, any closed interval $([a,b]=\bigcap_{i=1}^{\infty}(a - \frac{1}{i}, b + \frac{1}{i})).$\\ Most reasonable sets are Borel:\\
\([a,b)=\bigcap_{n=1}^{\infty}(a - \frac{1}{n}, b),  \{a\}=\bigcap_{n=1}^{\infty}(a - \frac{1}{n}, a + \frac{1}{n},   \QQ=\bigcup_{n=1}^{\infty}r_n,  I(irrationals)=\QQ^{c}.\)
\end{definition}

\begin{proposition} \label{P:OpenSetsBorel}
	Open sets are Borel.
	\begin{proof}
		If $G$ is open and $g \in G$, we can choose an $I_g$ with rational end points such that $g \in I_g \subset G$.  Since there are only countably many open intervals with rational end points, we may arrange the intervals $I_g$, $g \in G$ as a sequence of open intervals $\{I_n\}_{n=1}^\infty$.  Then $G = \bigcup_{n=1}^\infty I_n$ and so $G$ is a Borel set.
	\end{proof}
\end{proposition}

\begin{corollary} \label{C:ClosedSetsBorel}
	Closed sets are Borel sets.
	\begin{proof}
		They are complements of open sets
	\end{proof}
\end{corollary}

\begin{note}
	Two different collections of sets can give rise to the same $\sigma$-algebra.
\end{note}

\begin{example}
	Let
	\begin{align*}
		I &= \text{collection of open intervals in } \mathbb{R} \text{ and} \\
		\theta &= \text{collection of open sets in } \mathbb{R}.
	\end{align*}
	Then $I \subset \theta$ so $I \subset \sigma(\theta)$.  $\sigma(I)$ is the smallest $\sigma$-algebra containing $I$ so $\sigma(I) \subset \sigma(\theta)$.\\
	Open sets are Borel sets so $\theta \subset \sigma(I)$.  $\sigma(\theta)$ is the smallest $\sigma$-algebra containing $\theta$ so $\sigma(\theta) \subset \sigma(I)$. Hence $\sigma(\theta) = \sigma(I)$
\end{example}


\begin{definition} \label{D:Measure}
	If $\mathcal{F}$ is a $\sigma$-algebra on a set $\Omega$, then a \emph{measure on $\mathcal{F}$} is a function, $\mu$ such that:
	$$\mu : \mathcal{F} \rightarrow [0,\infty]$$
	satisfying:
	\begin{enumerate}
		\item $\mu(\emptyset) = 0$
		\item If $E_1,E_2,... \in \mathcal{F}$ and $E_i \cap E_j = \emptyset, i \not= j$, then
			$$\mu(\bigcup_{i=1}^\infty E_i) = \sum_{i=1}^\infty \mu(E_i)$$
	\end{enumerate}
\end{definition}

\begin{example} \label{E:Dirac}
	Let $\Omega = \text{any set},\, \mathcal{F} = \{\text{all subsets of }\Omega\}$.  Fix $x \in \Omega$, then for $E \in \mathcal{F}$ define
	$$\delta_x (E) = 
	\begin{cases}
		0, & \text{if } x \notin E\\
		1, & \text{if } x \in E
	\end{cases}$$
	We claim that $\delta_x$ is a measure on $\mathcal{F}$.
	\begin{proof}We prove the properties of measures one-by-one.
		\begin{enumerate}
			\item $\delta_x (\emptyset) = 0$
			\item If $E_1,E_2,... \in \mathcal{F}$ and $E_i \cap E_j = \emptyset, i \not= j$, then,
			\begin{description}
				 \item either $x \notin \bigcup_{i=1}^\infty E_i$ and hence $x \notin E_i$ for all $i$ so $$\delta_x (\bigcup_{i=1}^\infty) = 0 = \sum_{i=1}^\infty \delta_x(E_i)$$
				\item or $x \in \bigcup_{i=1}^\infty E_i$ so $x \in \text{exactly one} E_j$ and $\delta_x (E_i) = 0, \text{for} i \not= j$.  Then $$\delta_x(\bigcup_{i=1}^\infty E_i) = 1 = \delta_x (E_j) = \sum_{i=1}^\infty \delta_x (E_i)$$
			\end{description}
		\end{enumerate}
	\end{proof}
\end{example} 

\begin{note}
	If $c \in [0, \infty]$, then $c \delta_x$ is also a measure. ($\infty \cdot 0 = 0$)
\end{note}

\begin{example} \label{E:Counting}
	We define the discrete counting measure, $\gamma$, by
	$$\gamma (E) = \sum_{x \in E} 1 = \text{number of elements in $E$}$$
\end{example}	

\begin{proposition} \label{P:PropertiesMeasures}
	\emph{Properties of Measures}
	\begin{enumerate}
		\item If $A, B \in \mathcal{F}$ and $A \subset B$, then
			$$\mu(A) \leq \mu(B)$$
		\item If $A, B \in \mathcal{F}$, $A \subset B$ and $\mu(A) < \infty$, then
			$$\mu(B) - \mu(A) = \mu(B \setminus A)$$
		\item \emph{$\sigma$-subadditivty}. If $E_1, E_2, ... \in \mathcal{F}$, then
			$$\mu(\bigcup_{i=1}^\infty) \le \sum_{i=1}^\infty \mu(E_i)$$
		\item \emph{Continuity of measures}. If $E_1, E_2, ... \in \mathcal{F}$ and $E_1 \subset E_2 \subset ...$, then
			$$\mu(\bigcup_{i=1}^\infty E_i) = \lim_{n \to \infty} \mu (E_n)$$
		\item If $E_1, E_2, ... \in \mathcal{F}$, $E_1 \supset E_2 \supset ...$ and $\mu(E_1) < \infty$, then 
			$$\mu(\bigcap_{i=1}^\infty E_i) = \lim_{i \to \infty} \mu(E_i)$$
	\end{enumerate}
	\begin{proof} \quad
		\begin{enumerate}
			\item $\mu(B) = \mu(A) + \mu(B \setminus A)$ and $\mu(B \setminus A) \ge 0$, so $\mu(B) \ge \mu(A)$.
			\item Rearrange 1. and $\mu(A) < \infty$ so the sum makes sense.
			\item Let 
			$$F_i = E_i \setminus \bigcup_{i < j} E_j$$
			then $\bigcup_{i=1}^\infty F_i = \bigcup_{i=1}^\infty E_i$, $F_1, F_2, ... \in \mathcal{F}$, $F_i \cap F_j = \emptyset, i \ne j$ and $F_i \subset E_i$ for all $i$.\\
			So
			$$\mu(\bigcup_{i=1}^\infty E_i) = \mu(\bigcup_{i=1}^\infty F_i) = \sum_{i=1}^\infty \mu(F_i) \le \sum_{i=1}^\infty \mu(E_i)$$
			\item Let 
			$$F_i = E_i \setminus \bigcup_{i < j} E_j$$
			then
			\begin{align*}
				\mu(\bigcup_{i=1}^\infty E_i)	= \mu(\bigcup_{i=1}^\infty F_i) %\\
				= \sum_{i=1}^\infty \mu(F_i) %\\
				= \lim_{n \to \infty} \sum_{i=1}^n \mu(F_i) %\\
				&= \lim_{n \to \infty} \mu(\bigcup_{i=1}^n F_i) \\
				&= \lim_{n \to \infty} \mu(E_n)
			\end{align*}
			\item $$\mu(E_1) = \mu(E_1 \setminus \bigcap_{i=1}^\infty E_i) + \mu(\bigcap_{i=1}^\infty E_i)$$
			So
			\begin{align*}
				\mu(\bigcap_{i=1}^\infty E_i) &= \mu(E_1) - \mu(E_1 \setminus \bigcap_{i=1}^\infty E_i)\\
				&= \mu(E_1) - \mu(\bigcup_{i=1}^\infty E_1 \setminus E_i)\\
				&= \mu(E_1) - \lim_{n \to \infty} \mu(E_1 \setminus E_n)\\
				&= \mu(E_1) - \mu(E_1) + \lim_{n \to \infty} \mu(E_n)
			\end{align*}
		\end{enumerate}
	\end{proof}
\end{proposition}

$\lambda$, \emph{Lebesgue measure}, will be our means of defining a concept for length, area, volume etc. of a set. \\
On $\mathbb{R}$ we clearly desire $\lambda_1 ((a,b)) = b - a$.\\
On $\mathbb{R}^2$ we clearly desire $\lambda_2 ((a,b) \times (c,d)) = (b - a)(d - c)$.
And so on. \\[11pt]
On $\mathbb{R}$, if a set $A$ is contained in $\bigcup_{i=1}^\infty(a_i,b_i)$ we must have by $\sigma$-subadditivity:
$$\lambda(A) \le \sum_{i=1}^\infty \lambda(a_i,b_i) \le \sum_{i=1}^\infty (b_i - a_i)$$
which motivates us to define:
$$\lambda^*(A) = \inf\{\sum_{i=1}^\infty(b_i - a_i) : A \subset \bigcup_{i=1}^\infty(a_i,b_i)\}$$
We plan to find a $\sigma$-algebra containing the open intervals (and hence the Borel sets) on which $\lambda^*$ is a measure.\\
We must also show $\lambda^*((a,b)) = b - a$.

%Outer measure

\section{Outer Measure}

\begin{definition} \label{D:OuterMeasure}
	An \emph{outer measure} on a set $\Omega$ is a function:
	$$\mu^* : \{\text{All subsets of $\Omega$}\} \to [0,\infty]$$
	such that:
	\begin{enumerate}
		\item $\mu^*(\emptyset) = 0$
		\item \emph{Monotonicity}. If $A \subset B$, then $\mu^*(A) \le \mu^*(B)$.
		\item \emph{$\sigma$-subadditivity}. $\mu(\bigcup_{i=1}^\infty B_i) \le \sum_{i=1}^\infty \mu^*(B_i)$
	\end{enumerate}
\end{definition}

\begin{definition}
	Let $\mathcal{A}$ be a family of subsets of $\Omega$. Now define a function $\phi : \mathcal{A} \to [0,\infty]$.  Let $B$ be an arbitrary subset of $\Omega$ and define:
	$$\mu^*(A) = \inf\{\sum_{i=1}^\infty \phi(A_i) : B \subset \bigcup_{i=1}^\infty A_i, \text{ and } A_1, A_2, ... \in \mathcal{A} \}$$
	Further, define $\mu^*(\emptyset) = 0$ and $\mu^*(B) = \infty$ if no such $A_i$ (i.e. no such cover) exist(s).
\end{definition}

\begin{lemma} \label{L:OuterMeasure}
	$\mu^*$ is an outer measure.
	\begin{proof} \quad
		\begin{enumerate}
		\item from definition
		\item from definition
		\item Consider a set $B_j$ and cover $B_j$ by sets $A_{i}^{(j)}$ in $\mathcal{A}$ such that $$B_j \subset \bigcup_{i=1}^\infty A_{i}^{(j)} \text{ and } \sum_{i=1}^\infty \phi (A_{i}^{(j)}) \le \mu^*(B_j) + \frac{\epsilon}{2^j}$$
		then
		$$\mu^*(\bigcup_{i=1}^\infty B_j) \le \sum_{j=1}^\infty \sum_{i=1}^\infty \phi(A_{i}^{(j)}) \le \sum_{i=1}^\infty \mu^*(B_j) + \epsilon$$
		\end{enumerate}
	\end{proof}
\end{lemma}

\begin{definition} \label{D:measurability}
	If $\mu^*$ is an outer measure on a set $\Omega$, we say that a set $A \subset \Omega$ is \emph{$\mu^*$ measurable} if, for any $T \subset \Omega$:
	$$\mu^*(T \cap A) + \mu^*(T \setminus A) = \mu^*(T)$$
\end{definition}

\begin{theorem} \label{T:Caratheodory}
	If $\mu^*$ is an outer measure on $\Omega$, then the family of $\mu^*$ measurable sets, $\mathcal{F}(\mu^*)$, is a $\sigma$-algebra and $\mu^*$ is a measure on $\mathcal{F}(\mu^*)$.
	\begin{proof}
		We first show that $\mathcal{F}(\mu^*)$ is an algebra:
		\begin{enumerate}
			\item If $A \in \mathcal{F}(\mu^*)$, then $\Omega \setminus A \in \mathcal{F}(\mu^*)$
			\begin{proof}
					For any $T \subset \Omega$
					\begin{align*}
						\mu^*(T) &= \mu^*(T \cap A) + \mu^*(T \setminus A)\\
						&= \mu^*(T \setminus (\Omega \setminus A)) + \mu^*(T \cap (\Omega \setminus A))\\
					\end{align*}
			\end{proof}
			\item $\emptyset \in \mathcal{F}(\mu^*)$ and $\Omega \in \mathcal{F}(\mu^*)$
			\begin{proof}
					For any $T \subset \Omega$
					\begin{align*}
						\mu^*(T) &= \mu^*(T \cap \emptyset) + \mu^*(T \setminus \emptyset)\\
						&= 0 + \mu^*(T)\\
						&= \mu^*(T)
					\end{align*}
					By 1., $\Omega \in \mathcal{F}(\mu^*)$.
			\end{proof}
			\item If $A, B \in \mathcal{F}(\mu^*)$, then $A \cup B \in \mathcal{F}(\mu^*)$
			\begin{proof}
					Let $A, B \in \mathcal{F}(\mu^*)$ and let $T \subset \Omega$ be an arbitrary set. $A$ is $\mu^*$ measurable, so
					$$\mu^*(T) = \mu^*(T \cap A) + \mu^*(T \setminus A)$$
					We now test the measurability of $B$ with $T \cap A$
					\begin{align*}
						\mu^*(T \cap A) &= \mu^*(T \cap A \cap B) + \mu^*((T \cap A) \setminus B)\\
						\text{so } \mu^*(T) &= \mu^*(T \cap A \cap B) + \mu^*((T \cap A) \setminus B) + \mu^*(T \setminus A)\\
						&\ge \mu^*(T \cap A \cap B) + \mu^*(((T \cap A) \setminus B) \cup (T \setminus A))\\
						&\ge \mu^*(T \cap (A \cap B)) + \mu^*(T \setminus (A \cap B))
					\end{align*}
					Since $(((T \cap A) \setminus B) \cup (T \setminus A)) \supset (T \setminus (A \cap B))$ (monotonicity).\\
					Now, by the subadditivity of outer measures,
					$$\mu^*(T) \le \mu^*(T \cap (A \cap B)) + \mu^*(T \setminus (A \cap B))$$
					and hence
					$$\mu^*(T) = \mu^*(T \cap (A \cap B)) + \mu^*(T \setminus (A \cap B))$$
					So $A \cap B$ is $\mu^*$ measurable, for $A, B \in \mathcal{F}(\mu^*)$.  By \emph{De Morgan's Laws}, $A \cup B = \Omega \setminus ((\Omega \setminus A) \cap (\Omega \setminus B))$ so, by 2., $A \cup B \in \mathcal{F}(\mu^*)$.
			\end{proof}
		\end{enumerate}
		So $\mathcal{F}(\mu^*)$ is an algebra.  We must now prove that $\mathcal{F}(\mu^*)$ is a $\sigma$-algebra.\\[11pt]
		Let $F_1, ... , F_n \in \mathcal{F}(\mu^*)$ be disjoint sets, then, since $\mathcal{F}(\mu^*)$ is an algebra, $\bigcup_{i=1}^n F_i \, \text{and} \, \bigcap_{i=1}^n F_i \in \mathcal{F}(\mu^*)$.\\
		We claim $\mu^*(T \cap \bigcup_{i=1}^n F_i) = \sum_{i=1}^n \mu^*(T \cap F_i)$ for all $n$.
		\begin{proof}
			Let $n = 1$, then trivially $\mu^*(T \cap F_1) = \mu^*(T \cap F_1)$.\\
			Assume our claim holds for some $n \ge 1$, then consider
			\begin{align*}
				\mu^*(T \cap \bigcup_{i=1}^{n+1} F_i) &= \mu^*((T \cap \bigcup_{i=1}^{n+1} F_i) \cap F_{n+1}) + \mu^*((T \cap \bigcup_{i=1}^{n+1} F_i) \setminus F_{n+1})\\
				&= \mu^*(T \cap F_{n+1}) + \mu^*(T \cap \bigcup_{i=1}^n F_i)\\
				&= \mu^*(T \cap F_{n+1}) + \sum_{i=1}^n \mu^*(T \cap F_i)\\
				&= \sum_{i=1}^{n+1} \mu^*(T \cap F_i)
			\end{align*}
			So our claim holds inductively for all $n$.
		\end{proof}
		Now let $E_1, E_2, ... \in \mathcal{F}(\mu^*)$ be arbitrary sets and define
		$$F_i = E_i \setminus \bigcup_{j < i} E_j$$
		So that $F_i \cap F_j = \emptyset, i \ne j$ and, since $\mathcal{F}(\mu^*)$ is an algebra, $F_i \in \mathcal{F}(\mu^*)$ for all $i$. Also note that
		$$\bigcup_{i=1}^n E_i = \bigcup_{i=1}^n F_i  \text{ for all $n$, so } \bigcup_{i=1}^\infty E_i = \bigcup_{i=1}^\infty F_i$$
		Now let $T \subset \Omega$ be any set and recall $\bigcup_{i=1}^n F_i \in \mathcal{F}(\mu^*)$, so
		\begin{align*}
			\mu^*(T) &= \mu^*(T \cap \bigcup_{i=1}^n F_i) + \mu^*(T \setminus \bigcup_{i=1}^n F_i)\\
			&\ge \mu^*(T \cap \bigcup_{i=1}^n F_i) + \mu^*(T \setminus \bigcup_{i=1}^\infty F_i)\\
			&= \sum_{i=1}^n \mu^*(T \cap F_i) + \mu^*(T \setminus \bigcup_{i=1}^\infty F_i)\\
			& \xrightarrow{\text{as } n \to \infty} \sum_{i=1}^\infty \mu^*(T \cap F_i) + \mu^*(T \setminus \bigcup_{i=1}^\infty F_i)\\
			& \ge \mu^*(T \cap \bigcup_{i=1}^\infty F_i) + \mu^*(T \setminus \bigcup_{i=1}^\infty F_i)
		\end{align*}
		But $\mu^*$ is subadditive so
		$$\mu^*(T) \le \mu^*(T \cap \bigcup_{i=1}^\infty F_i) + \mu^*(T \setminus \bigcup_{i=1}^\infty F_i)$$
		Consequently
		\begin{align*}
			\mu^*(T) &= \mu^*(T \cap \bigcup_{i=1}^\infty F_i) + \mu^*(T \setminus \bigcup_{i=1}^\infty F_i)\\
			&= \mu^*(T \cap \bigcup_{i=1}^\infty E_i) + \mu^*(T \setminus \bigcup_{i=1}^\infty E_i)
		\end{align*}
		So $\bigcup_{i=1}^\infty E_i \in \mathcal{F}(\mu^*)$ and $\mathcal{F}(\mu^*)$ is a $\sigma$-algebra.	
	\end{proof}
\end{theorem}

\begin{definition} \label{D:MeasurableTerminology}
	If we restrict $\mu^*$ to $\mathcal{F}(\mu^*)$, then we replace $\mu^*$ by $\mu$ and simply say ``the measure $\mu$".
\end{definition}

%Lebesgue Measure

\section{Lebesgue Measure}

\begin{definition} \label{D:LebesgueOuterMeasure}
	The \emph{Lebesgue outer measure on $\mathbb{R}$} is defined as
	$$\lambda^*(A) = \inf\{\sum_{i=1}^\infty(b_i - a_i) : A \subset \bigcup_{i=1}^\infty(a_i,b_i)\}$$
\end{definition}

\begin{note}
	We could use any interval type to define $\lambda^*$.
	\begin{proof}
		\begin{enumerate}
			\item If $A \subset \bigcup_{i=1}^\infty (a_i, b_i)$, then $A \subset \bigcup_{i=1}^\infty [a_i, b_i]$ and so
			$$\inf\{\sum_{i=1}^\infty(b_i - a_i) : A \subset \bigcup_{i=1}^\infty[a_i,b_i]\} \le \inf\{\sum_{i=1}^\infty(b_i - a_i) : A \subset \bigcup_{i=1}^\infty(a_i,b_i)\}$$
			\item Let $\epsilon > 0$.  If $A \subset \bigcup_{i=1}^\infty [a_i, b_i]$, then $A \subset \bigcup_{i=1}^\infty (a_i - \frac{\epsilon}{2^i}, b_i + \frac{\epsilon}{2^i})$ and
			$$\sum_{i=1}^\infty ((b_i + \frac{\epsilon}{2^i}) - (a_i - \frac{\epsilon}{2^i})) = 2\epsilon + \sum_{i=1}^\infty (b_i - a_i)$$
			So
			$$\inf\{\sum_{i=1}^\infty(b_i - a_i) : A \subset \bigcup_{i=1}^\infty (a_i, b_i)\} \le 2\epsilon + \inf\{\sum_{i=1}^\infty(b_i - a_i) : A \subset \bigcup_{i=1}^\infty [a_i, b_i]\} $$
		\end{enumerate}
		Combining 1. and 2. yields equality.
	\end{proof}
\end{note}

\begin{lemma} \label{L:leb-a}
	If $[a,b] \subset \bigcup_{i=1}^\infty(a_i,b_i)$, then $b - a \le \sum_{i=1}^\infty (b_i - a_i)$
	\begin{proof}
		By Heine-Borel theorem, if a closed interval is contained in an union of open intervals, then there exists a finite subcover of the closed interval.  In our case there exists a finite $n$ such that
		$$[a,b] \subset \bigcup_{i=1}^n (a_i, b_i)$$
		So we need only show that for such an $[a,b] \subset \mathbb{R}$, $b - a \le \sum_{i=1}^n (b_i - a_i)$.\\
		Result holds for $n=1$. Assume result holds for some finite $n \ge 1$. For the case $n + 1$, we may assume $a_{n+1} \le a_i$ for all $i$ and $a_{n+1} < a$.\\
		\begin{enumerate}
			\item If $b_{n+1} > b$, then
			$$b - a \le b_{n+1} - a_{n+1} \le \sum_{i=1}^{n+1} (b_i - a_i)$$
			\item If $b_{n+1} < b$ (and $b_{n+1} > a$), then $[b_{n+1}, b]$ is covered by $\bigcup_{i=1}^n (a_i, b_i)$, so by inductive hypothesis
			\begin{align*}
				b - a &= (b - b_{n+1}) + (b_{n+1} - a)\\
				&\le \sum_{i=1}^n (b_i - a_i) + (b_{n+1} - a_{n+1})\\
				&= \sum_{i=1}^{n+1} (b_i - a_i)
			\end{align*}
		\end{enumerate}
		1. and 2. prove our claim inductively for $n+1$, so claim holds inductively for all $n$ and our lemma is proved.
	\end{proof}
\end{lemma}

\begin{lemma} \label{L:b-a}
	$\lambda^*(a,b) = \lambda^*[a,b] = b - a$
	\begin{proof}
		Note, by Definition \ref{D:LebesgueOuterMeasure}
		$$\lambda^*[a,b] =  \inf\{\sum_{i=1}^\infty(b_i - a_i) : [a,b] \subset \bigcup_{i=1}^\infty(a_i,b_i)\}$$
		Now $[a, b] \subset (a - \epsilon, b + \epsilon)$ for all $\epsilon > 0$ so
		$$\lambda^*[a,b] \le b - a + 2\epsilon$$
		and by Lemma \ref{L:leb-a} we may deduce
		$$\lambda^*[a,b] = b - a$$
		Furthermore
		$$b - a  - 2\epsilon \le \lambda^*[a+\epsilon, b - \epsilon] \le \lambda^*(a,b) \le \lambda^*[a,b] = b - a$$
		So $\lambda^*(a,b) = b - a$ also.
	\end{proof}
\end{lemma}

\end{document}