\documentclass[12pt]{article}
\usepackage{amsfonts, amsthm, amsmath}
\usepackage{verbatim}
\usepackage{enumerate}
\usepackage{graphicx}
\usepackage{amssymb}
\usepackage{epstopdf}
\usepackage{amsthm}
\usepackage{mathtools}
\usepackage{pb-diagram}
\usepackage{hyperref}
\usepackage{cancel}

\setlength{\textwidth}{6.5in}
\setlength{\oddsidemargin}{0in}
\setlength{\textheight}{9.5in}
\setlength{\topmargin}{0in}
\setlength{\headheight}{0in}
\setlength{\headsep}{0in}
\setlength{\parskip}{0pt}
\setlength{\parindent}{0pt}

\def\CC{\mathbb{C}}
\def\MM{\mathbb{M}}
\def\FF{\mathbb{F}}
\def\PP{\mathbb{P}}
\def\QQ{\mathbb{Q}}
\def\RR{\mathbb{R}}
\def\ZZ{\mathbb{Z}}
\def\gotha{\mathfrak{a}}
\def\gothb{\mathfrak{b}}
\def\gothm{\mathfrak{m}}
\def\gotho{\mathfrak{o}}
\def\gothp{\mathfrak{p}}
\def\gothq{\mathfrak{q}}
\DeclareMathOperator{\disc}{Disc}
\DeclareMathOperator{\Gal}{Gal}
\DeclareMathOperator{\GL}{GL}
\DeclareMathOperator{\Hom}{Hom}
\DeclareMathOperator{\Norm}{Norm}
\DeclareMathOperator{\Trace}{Trace}
\DeclareMathOperator{\Cl}{Cl}

\def\head#1{\medskip \noindent \textbf{#1}.}

\newtheorem{theorem}{Theorem}[section]
\newtheorem{lemma}[theorem]{Lemma}
\newtheorem{definition}{Definition}[section]
\newtheorem{example}{Example}[section]
\newtheorem{proposition}{Proposition}[section]
\newtheorem{corollary}{Corollary}[section]
\newtheorem*{note}{Note}
\newtheorem*{remark}{Remark}
\newtheorem*{claim}{Claim}


\begin{document}
\title{Algebraic Number Theory\\ 3704}
\author{Dr H Wilton}
\date{Jan 2012}
\maketitle

\tableofcontents
\setcounter{tocdepth}{4}
\newpage

\section{Introduction}
An Algebraic number is the root of a polynomial. \\
eg. $\alpha = \sqrt{2}, \; 15\sqrt[7]{3}, \; 2 + i , \dots$ such that $f(\alpha ) = 0$ where $f \in \ZZ [x]$ or $\QQ [x]$\\
An Algebraic number field\\
$\QQ (\sqrt{2}) = \{a + b\sqrt{2} | a,b \in \QQ \}$ is the smallest subfield of $\CC$ containing both $\QQ$ and $\sqrt{2}$\\ also $\QQ (i + \sqrt{2})$ for example\\
$K$ is an algebraic number field $\underbrace{\sigma }_{\mathclap{\text{algebraic integers}}} \subseteq K$ eg. $\ZZ [x] \subseteq \QQ [x]$\\
Typical questions to ask about $\sigma$
\begin{enumerate}[(i)]
\item Does $\sigma$ have unique factorisation?
\item Is $\sigma$ a PID?
\item If not then how close is $\sigma$ to being a PID?
\item How does a prime $p$ factorise in $\sigma$?\\
eg. in $\ZZ [i], \; 5=(\sqrt{2} +1)(\sqrt{2} -1)$ but $7$ doesn't factorise.
\item What are the units of $\sigma$?\\
eg. $(\sqrt{2} +1)(\sqrt{2} -1) = 1$ in $\ZZ (\sqrt{2})$ but in $\ZZ (\sqrt{-5})$ only $1,-1$ are units.
\end{enumerate}

\section{Fields}
\subsection{Background material}
Rings - commutative with 1\\
$K$ - a field\\
Rings of interest
\begin{enumerate}
\item $\ZZ$
\item $K[x] = \{f(x) = \sum_{i=0}^{n}a_{i}x^{i} \; | \; a^i \in K\}$
\end{enumerate}
Members of the ring:
\begin{enumerate}[(i)]
\item units - irreducible elements
\item reductible elements $f = gh, \; g,h$ non units
\item irreductible elements, everything else
\end{enumerate}
eg. Units of $K[x] = K^*$\\

\textbf{Criteria for irreductibility of $f \in \QQ[x]$}
\begin{enumerate}[(i)]
\item Gauss lemma: If $f$ is irreducible in $\ZZ [x]$ then $f$ is irreducible in $\QQ [x]$\\
Corollary: if $f$ is monic of degree 2 or 3 then if $f$ is reductible it has a root in $\ZZ$ which has to divide the constant term of $f$. eg. $x^3 + x +1$ \\
\item Eissenstein Criterion: $f(x) = \sum_{i=0}^{n}a_{i}x^{i}$ if there is $p$ a prime such that:
\begin{enumerate}
\item $p\mid a_i, \; i < n$
\item $p\nmid a_n$
\item $p\nmid a_0$
\end{enumerate}
then $f$ is irreducible. eg. for: $x^2 + 4x + 2$
\item Reduction mod $p$\\
if $f \in \ZZ [x]$ denote the map:
\begin{align*}
\ZZ[x] & \rightarrow (\ZZ / p)[x] \\
by \quad f & \rightarrow f
\end{align*}
if the degree of the polynomial doesn't go down and deg $f$ = deg $f$ and f is irreductible in $(\ZZ / p)[x]$, then $f$ is irreductible in $\ZZ [x]$.\\
\end{enumerate}
Also note that $\underbrace{f(x)}_{\mathclap{\in \ZZ [x]}}$ s irreductable iff $f(x+a)$ is irreductable where $a \in \ZZ$\\
\begin{definition}[Euclid's algorithm]
If $f,g \in K[x]$ then we can write 
\[f(x) = h(x)g(x) + r(x) \qquad deg(r) < deg(g)\]
\[hcf(f,g) = hcf(g,r)= \dots \]
\end{definition}

\begin{definition} A ring with a euclidean algorithm is called a euclidean domain eg. $\ZZ [x], \; K[x]$ where a euclidean algorithm assigns each menber of the ring a degree $deg:R \rightarrow \mathbb{N}$
\end{definition}

\begin{definition}[Ideal]
$R$-ring, $I\subseteq R, \; I \neq \phi$ is called an ideal if:
\begin{enumerate}[(i)]
\item $x,y \in I \Rightarrow x+y \in I$
\item $x \in I, \; \lambda \in R \Rightarrow \lambda x \in I$
\end{enumerate}
\[eg.\; x \in R \text{ then }(x) = \{\lambda x \; | \; \lambda \in R \} \;-\text{ Principal ideal.}\]
\[Also \; (x_1 , \dots , x_n ) = \left\{ \sum_{i=1}^{n}\lambda_ix^i \; | \; \lambda_i \in R\right\} \; eg. \;\underbrace{(4,6)}_{\mathclap{\subset \ZZ}} = (hcf(4,6)) = (2)\]
\end{definition}
\begin{definition} If every ideal in $R$ is a principal ideal then $R$ is a principal ideal domain.
\end{definition}

\begin{theorem}[Euclidean rings are PID]\quad\\
\begin{proof}
$I \subseteq R$, ideal. Take $x \in I/0$ of minimal degree. Let $y \in I$ then 
\[y=gx + r, \quad deg(r)<deg(x), \quad r \in I \quad \Rightarrow r=0\]
\end{proof}
\end{theorem}


\begin{definition}[Maximal ideal]
an ideal $I \subseteq R$ is maximal if for any ideal $J$ with $I \subseteq J \subseteq R$ either $I =J$ or $J=R$.
\end{definition}

\begin{example}
maximal ideals in $K[x]$ are all of the form $(f)$ where f is and irreductable polynomial. for $(g)$ if $g=hk$ then $(g) \subsetneq (h)$. 
\end{example}
\begin{definition}
Let $I \subseteq R$ be an ideal then $(I,+) \subseteq (R,+)$ is a subgroup. We can consider the group 
\[R/I = \{x+I\;|\;x \in R\}\]
\begin{align*}
(x+I) +(y+I) &= (x+y) + I\\
(x+I)(y+I) &= xy + I
\end{align*}
$R/I$ is the quotient of R by I.
\end{definition}

\begin{definition}
If $R,S$ are rings, $\phi:R \rightarrow S$ is a ring homomorphism if:
\begin{enumerate}[(i)]
\item $\phi (a+b) = \phi (a) + \phi (b)$
\item $\phi (ab) = \phi (a) \phi (b)$
\item $\phi (1) = 1$
\end{enumerate}
\end{definition}

\begin{lemma}\label{field ideal}
If $K$ is a field and $I \subseteq K$ is an ideal then $I = \{ 0 \}$ or $I=\{ K\}$
\begin{proof}
If $x \in I/ \{0\}$ and $y \in K$ be arbitrary. Then 
\[\underbrace{(y x^{-1})x}_{\mathclap{= y}} \in I\]
\end{proof}
\end{lemma}

\begin{corollary}
If $\phi:K \rightarrow R$ is a ring homomorphism where $K$ is a field and $R$ is a ring, then $\phi$ is injective.
\begin{proof}
\[\phi(1) = 1_{R} \; so \; 1 \notin ker(\phi ) \Rightarrow ker(\phi ) \neq K  \therefore ker(\phi) = \{0 \}\]
\end{proof}
\end{corollary}

\begin{theorem}
An ideal $i \subset R$ is maximal iff $R/I$ is a field.
\begin{proof} \quad \\
($\Leftarrow $) 
Let 
\[\phi :R \rightarrow R/I, \quad \phi: x \mapsto x +I \qquad\text{be the quotient homomorphism}\]
Suppose $I \subseteq J \subseteq R$ then
\[ \phi (J) \subseteq R/I \]
is an ideal. By the lemma \ref{field ideal}
\begin{align*}
\phi (J) = \{0 \} &\Rightarrow J= I\\
\phi (J) = R/I  &\Rightarrow J= R 
\end{align*}
($\Rightarrow $)
Suppose $I \subseteq R$ is maximal and consider $x \in R/I$. We need to show that $x+I \in R/I$ has a multiplicative inverse. The ideal generated by $x$ and $I$  is $R$. $1 \in R$ so there is $y \in R$ and $\xi \in I$ such that
\[xy + \xi = 1 \Rightarrow 1 \in xy+I = (x+I)(y+I)\]
$x+I$ has a multiplicative inverse, hence $R/I$ is a field. 
\end{proof}
\end{theorem} 

\subsection{Field Extentions}

\begin{definition}
if $K, \; L$ are fields and $K \subseteq L$ then $K$ is a subfield of $L$ and $L$ is an extention of $K$
\end{definition}

\begin{example}
\[K= \QQ , \quad L=\QQ (\sqrt{2}) = \{a + b\sqrt{2} \; | \;a,b \in \QQ\}\]
\[K \subseteq L \subseteq \CC\]
\end{example}

\begin{definition}
An element $\alpha \in L$ is algebraic over $K$ if:
\[ \exists f(x) \in K[x] \quad \text{ such that}\quad f(\alpha )=0.\]
\end{definition}
Usually $k= \QQ$

\begin{definition}
The ring generated by $K$ and $\alpha \in L$ is denoted $K[\alpha ]$:
\[  K[\alpha ] = \{f(\alpha ) \; |\; f \in K[x] \}.\]
\end{definition}

\begin{definition}
The field generated by $K$ and $\alpha \in L$ is denoted $K(\alpha )$:
\[  K(\alpha ) = \left\{ \frac{f(\alpha )}{g(\alpha )} \; |\; f,g \in K[x], \; g(\alpha ) \neq 0 \right\}.\]
\end{definition}

\begin{definition}
\[  I(\alpha ) = \{ f \in K[x] \; | \; f(\alpha ) = 0 \}.\]
\end{definition}

\begin{lemma} $I(\alpha )$ is an ideal.
\begin{proof}$ f,g \in I(\alpha ) $
\[(f+g)(\alpha ) = f(\alpha) +  g(\alpha) = 0+0\]
$ f \in I(\alpha ), \; g \in K[x] $
\[(fg)(\alpha ) = f(\alpha )g(\alpha ) = 0 \cdot g(\alpha ) =0 \]
\end{proof}
\end{lemma}

\begin{definition}
$K[x]$ is a PID
\[I(\alpha ) = (M)\]
$M$ is well defined up to multiplication by $\lambda \in K^*$, because it's minimum degree element of $I(\alpha )$
\end{definition}

\begin{definition}
The minimal polynomial of $\alpha $, $M_{\alpha}$, is the unique monic polynomial such that
\[I(\alpha ) = M_{\alpha }\]
\end{definition}

\begin{example}$\alpha = \sqrt{2},\quad k=\QQ$ 
\[M_{\alpha}(x) = x^2 - 2 \]
\end{example}
\begin{lemma}
$I(x)$ is maximal or equlivlently, $M_{\alpha } $ is irreductable. 
\begin{proof}
suppose $M_{\alpha}$ is reductable then:
\begin{align*}
M_{\alpha}(x) &= a(x)b(x)\\
M_{\alpha}(\alpha ) &= a(\alpha )b(\alpha )= 0
\end{align*}
Without loss of generality
\[a(\alpha ) = 0 \Rightarrow a \in I(\alpha ) = (M_{\alpha })\]
So $M_{\alpha} | a$ and $deg(a) = deg(M_{\alpha})$ so $b(x)$ is constant. $b(x)$ is a unit in $K[x]$ and so $M_{\alpha}$ is irreductable. 
\end{proof}
\end{lemma}

\begin{lemma}
A polynomial $M$ is the minimal polynomial of $\alpha$ if:
\begin{enumerate}[(i)]
\item $M(\alpha ) = 0$
\item $M$ is monic 
\item $M$ is irreductable
\end{enumerate}
\begin{proof} \quad \\
($\Rightarrow$) already done\\
($\Leftarrow $)
\begin{align*}
(i) &\Rightarrow M \in I(\alpha ) = (M_{\alpha})\\
&\Rightarrow M_{\alpha} | M \; ie \; M= a \cdot M_{\alpha}
\end{align*}
by (iii) $a \in K^*$, compare coefficents and using the fact, (ii), that M is monic
\[x^m = ax^n \Rightarrow a = 1 \]
\end{proof}
\end{lemma}

we just proved that proved that $I(\alpha ) $ is a maximal ideal, so $K[x]/I(\alpha)$ is a field.

\begin{theorem}
Let $\alpha \in L$ be algebraic over $K$. Then: 
\begin{alignat*}{2}
\Phi:\quad &k[x]/I(\alpha ) & & \rightarrow k(\alpha )\\
&f + (M_{\alpha}) &  &\mapsto f(\alpha )
\end{alignat*}
is a field isomorphism and $K[\alpha] = k(\alpha)$.
\begin{proof}
first we need to check $\Phi$ is well defined. Suppose:
\begin{align*}
g \in f + (M_{\alpha}) &\Leftrightarrow [f-g \in (M_{\alpha})]
\shortintertext{then}
\Phi(g+ (M_{\alpha})) = g(\alpha ) &= f(\alpha ) = \Phi(f+ (M_{\alpha}))
\end{align*}
next we should check that $\Phi$ is a ring homomorphism. 
\[\Phi(1 + (M_{\alpha})) = 1\]
\[\Phi(f +g + (M_{\alpha})) = f+g = \Phi(f + (M_{\alpha})) + \Phi(g + (M_{\alpha}))\]
\[\Phi((f + (M_{\alpha})) \cdot (g + (M_{\alpha}))) = fg + f(M_{\alpha}) + g(M_{\alpha}) + (M_{\alpha})^2 = fg = \Phi(f + (M_{\alpha})) \cdot \Phi(g + (M_{\alpha}))\]
notice 
\[Im(\Phi) = K(\alpha)\]
but $k[x]/(M_{\alpha})$ is a field so $\Phi$ is injective, so we have
\[ k[x]/(M_{\alpha}) \cong \Phi(\underbrace{k[x]}_{\mathclap{\subseteq K}}/\underbrace{(M_{\alpha})}_{\mathclap{\ni \alpha}}) \subseteq K[\alpha] \subseteq K(\alpha).\]
Therefore by definition of $K(\alpha)$, 
\[\Phi(k[x]/(M_{\alpha})) = K[\alpha] = K(\alpha).\]
\end{proof}
\end{theorem}
It's normal to abuse notation and write $f$ for $f + I \in k[x]/I$
\begin{example}
$\alpha = \sqrt{2} + \sqrt{3}$ can talk about  $\QQ(\sqrt{2} + \sqrt{3})$
\[\alpha^{2} = 5 + 6\sqrt{6} \qquad (\alpha^{2} - 5)^2 = 24\]
so $\alpha$ is a root of $M = x^2 - 10x +1 = 0$, need to check $M$ is irreductable. Recall that a quartic can factor in two different ways
\begin{enumerate}[(i)]
\item $(quadratic) \times (quadratic)$
\item $(quadratic) \times (linear)$\\
\end{enumerate}
$(ii) \Rightarrow$ root is a factor of $1$
\[M(1) = -8 \qquad M(-1) = -8\]
$M$ does not have a linear factor\\
$(i)$
\begin{align*}
(x^2 + a x + b )(x^2 + cx + d) &= x^4 - 10x^2 +1 \\
(x^2 + a x + b )(x^2 + cx + d) &= x^4 (a+c)x^3 (ac + b + d)x^2 + (bc + ad)x + bd
\end{align*}
\begin{alignat*}{2}
&a+c=0 & \quad \Rightarrow \quad &a=-c\\
&bd=1 & \quad \Rightarrow \quad &b=d, \; b=\pm 1\\
&ac+ b+d=-10 & \quad \Rightarrow \quad &a^2 = 10 + 2b = 8\; or\; 12
\end{alignat*}
$8$ or $12$ not squares $\Rightarrow  x^4 - 10x +1$ is irreductable. So:
\begin{align*}
\QQ[x]/( x^4 - 10x +1)  &\cong \QQ(\sqrt{2} + \sqrt{3})\\
f  &\mapsto  f(\sqrt{2} + \sqrt{3})
\end{align*}
\end{example}

\subsection{Degrees of extension} 
$L\supset K$ if we we say $l_1+l_2$ and $kl$ are defined but $l_1l_2$ is not $\forall l_1,l_2 \in L, \; \forall k \in K$, this realises $L$ as a vector space over K.
\begin{definition}
The degree of $L$ over $K$ is just $dim(L)$ when $L$ is thought of as a vector space over $K$. It is denoted
\[ [L:K] \]
\end{definition}

\begin{example}
$\CC \supset \RR$
\[\CC = \{a+bi \;|\; a,b \in \RR \} \quad \{1,i\} \text{ form a basis over } \RR \]
So $[\CC:\RR] = 2$
\end{example}

\begin{example}
Let $f(x) =\sum_{i=0}^{d}a_ix^i$ be an irreductable polynomial over $K$
\[L=K[x]/(f) \supset K \qquad [L:K] = deg(f) \]
To show $B = \{1, x ,\dots , x^{d-1}\}$ is a basis:\\
Span: 
\[x^d = \frac{-1}{a_d}\sum_{i=0}^{d-1}a_ix^i \Rightarrow x^d \in span(B)\]
similarly $x^n \in span(B)$ for any $n \geq d$
\[x^n = x^{n-d}x^d = x^{n-d}\left( \frac{-1}{a_d}\sum_{i=0}^{d-1}a_ix^i \right) \; \text{is of degree } \leq n-1 \]
and so  $x^n \in span(B)$ by induction, but $\forall g \in K[x]/(f)$
\[ g \in span( \{1, x ,\dots , x^{n}\}) \in span( \{1, x ,\dots , x^{d-1}\}) = span( B)\]
Linear Independence:\\
suppose $g(x) = \sum_{i=0}^{d}b_ix^i = 0$. Then $g \in (f)$ but 
\[deg(g) \leq d-1 \leq d = deg(f) \Rightarrow g=0 \Rightarrow b_i =0 \forall i\]
therefore if $f=M_{\alpha}$ for some $\alpha$ algebraic over $K$ then
\[ [k(\alpha): K] = deg(M_{\alpha}) \]
\end{example}

\begin{proposition}
$\alpha$ is algebraic over $K$ iff $[K(\alpha):K] < \infty$
\begin{proof} \quad \\ 
$(\Rightarrow)$
 \[[k(\alpha):k] = deg(M_{\alpha}) < \infty \]
$(\Leftarrow)$ \\ 
suppose 
\[[k(\alpha):k] = d < \infty\] 
then $1,\alpha, \dots, \alpha^d$ is linearly independant $\Rightarrow$ there exists $a_i$ such that \[\sum_{i=0}^{d}a_i\alpha^i =0 \]
\end{proof}
\end{proposition}

\begin{theorem}[Tower Theorem]\label{Tower}
suppose $K \subseteq L \subseteq M$ then
\[ [M:K]=[M:L][L:K] \]
\begin{proof} \quad \\
Let $\{a_i\}$ be a basis for $L$ over $K$ \& let $\{b_i\}$ be a basis for $M$ over $L$
\begin{claim} $\{a_i , b_i \}$ is a basis for $M$ over $K$.\\
Span: Let $v \in M$, then $\exists \lambda_i \in L$ such that
\[v = \sum_j\lambda_jb_j \]
$\exists m_{i,j} \in K$ such that $\lambda_j= \sum_im_{i,j}a_i$ because $\lambda_i \in L$. So:
\[v = \sum_{i,j}m_{i,j}a_ib_j\]
Linear independance: suppose 
\[ \sum_{i,j}\underbrace{m_{i,j}}_{\mathclap{\in K}}a_ib_j = 0\]
Let $\lambda_j= \sum_im_{i,j}a_i$ then 
\[\sum_i\lambda_jb_j = 0 \Rightarrow \lambda_j = 0 \quad \forall j\]
So 
\[m_{i,j} = 0 \quad \forall i,j. \]
\end{claim}
\end{proof}
\end{theorem}

\begin{corollary}
Let $L$ be a field extention of $K, \; L\supseteq K$, and let $L^{alg} \subseteq L$ be the set of algebraic elements over $K$ f $L$. Then $L^{alg}$ is a field.
\begin{proof}
Let $\alpha, \; \beta \in L^{alg}$ then
\[ [K(\alpha , \beta):K] = [K(\alpha , \beta):K(\alpha)][K(\alpha):K] \]
Let $\theta < \alpha + \beta, \alpha\beta, \alpha - \beta, \frac{\alpha}{\beta} \in K(\alpha, \beta)$, now:
\[[K(\alpha , \beta):K] = [K(\alpha , \beta):K(\theta)][K(\theta):K] \]
therefore $[K(\theta):K]< \infty$ so $\theta \in L^{alg}$
\end{proof}
\end{corollary}

\begin{example}
what is the minimal polynomial of $\sqrt{3}$ over $\QQ(\sqrt{2})$? Hopefully it's still $x^2 -3$\\
note $\sqrt{2}, \; \sqrt{3} \in \QQ(\sqrt{2} + \sqrt{3})  = \QQ(\sqrt{2} , \sqrt{3})$
\[ (\sqrt{2} + \sqrt{3}) ^3 =  11\sqrt{2} + 9\sqrt{3}\]
\[(\sqrt{2} + \sqrt{3}) ^3 - 9(\sqrt{2} + \sqrt{3}) =  11\sqrt{2} + 9\sqrt{3} - 9 (\sqrt{2} + \sqrt{3}) = 2\sqrt{2}\quad so \; \sqrt{2}, \sqrt{3} \in \QQ(\sqrt{2} + \sqrt{3})\]
\[ [\QQ(\sqrt{2} + \sqrt{3}):\QQ] = deg(x^4 - 10x^2 +1) = 4\]
Therefore
\[4= [\QQ(\sqrt{2}, \sqrt{3}):\QQ(\sqrt{2})][\QQ(\sqrt{2}), \QQ] \]
by theorem \ref{Tower}
\[ [\QQ(\sqrt{2}, \sqrt{3}):\QQ(\sqrt{2})] = 2 \]
\end{example}

\begin{theorem}[Galois separability theorem]\label{sep}
$K \subseteq \CC$ and $f \in K[x]$ irreductable. Then $f$ does not have repeated roots in $\CC$. 
\begin{proof}
suppose $\alpha$ is a repeated root. Then
\[f(\alpha) = (x-\alpha)^2g(x) \quad g \in \CC[x]\]
\[ f'(\alpha) = (x-\alpha)^2g'(x)+ 2(x-\alpha)g(x)\]
So $f'(\alpha) =0$ then $f' \in I(\alpha) = (f)$, but 
\[deg(f')<deg(f) \Rightarrow f' = 0\]
Therefore $f$ is constant, a contradiction.
\end{proof}
\end{theorem}
\begin{note}
This doesn't work over finite fields eg. $\mathbb{F}_{p}$:
\[f=x^p -\alpha, \qquad f' = px^{p-1} =0\]
\end{note}

\begin{theorem}[Primitive element theorem]
suppose $K \subseteq L \subseteq \CC$ and $[L:K]<\infty$. Then $\exists \theta \in L$  such that
\[L = K(\theta) \]
\begin{proof}
Let $\{ 1 , \gamma_1 , \dots , \gamma_{d-1}\}$ be a basis for $L$ over $K$. Then 
\[ L = K(\gamma_1 , \dots , \gamma_{d-1}) = K(\gamma_1 , \dots , \gamma_{d-2})K(\gamma_{d-1}) \]
By induction on $d$ we may assume that $k(\gamma_1 , \dots , \gamma_{d-2}) = K(\alpha)$. Let $\gamma_{d-1} = \beta$, now $L= K(\alpha, \beta)$. Let $p=M_{\alpha} , \; q= M_{\beta}$ and let
\begin{alignat*}{2}
\alpha &= \alpha_1 , \dots, \alpha_{m}&\quad  &\text{ be the roots of $p$ and}\\
\beta &= \beta_1 , \dots, \beta_{n} &\quad &\text{ be the roots of $q$}
\end{alignat*}
Choose $c$ such that 
\[ \alpha_i + c\beta_j \neq \alpha + c\beta \quad unless \; i=j=1\]
To choose $c$ we use: 
\begin{enumerate}[(i)] 
\item $L$ is infinite 
\item we have finitely many $C$'s to avoid 
\item Galois separability theorem \ref{sep} $\Rightarrow \alpha_i = \alpha_{i'} \Rightarrow i=i'$ and $\Rightarrow \beta_i = \beta_{i'} \Rightarrow i=i'$
\end{enumerate} 
Let $\theta = \alpha + c\beta$ we need to prove that
\[ K(\theta) = k(\alpha ,\beta) \]
\begin{claim}
\[\beta \in K(\theta) \; \Rightarrow \; \alpha= \theta - c\beta \in K(\theta) \; \Rightarrow \; K(\alpha , \beta) \subseteq K(\theta) \subseteq K(\alpha , \beta) \]
Define $r(x) \in K(\theta)[x]$ by 
\[r(x) = p(\theta - cx)\]
Then 
\[r(\beta) = p(\theta - c\beta) = p(\alpha) = 0\]
on the other hand, $r(\beta_j) = p(\theta - c\beta_j) =0$ for $j \geq 2$
\begin{alignat*}{2}
&\Leftrightarrow \theta - c\beta_j = \alpha_i &\quad &\text{for some $i$}\\
&\Leftrightarrow \alpha + c\beta = \alpha_i + c\beta_j &\quad &\text{which never happens by choice of $c$}
\end{alignat*}
Now $\beta$ satisfies two polynomials over $K(\theta)$:
\[q(\beta) = 0 \quad \& \quad r(\beta)=0 \]
We have just seen that $\beta$ is the only root that $q$ and $L$ have in common. Let $M$ be the minimum polynomial of $\beta$ over $K(\theta)$
\[M|q \quad and \quad M|r\]
So and root of $M$ is a root of $q$ and $r$ the only root of $M$ is $\beta$ so $M=(x-\beta)^d$. $d = 1$ by Galois separability theorem \ref{sep} 
\[\Rightarrow M = x-\beta \Rightarrow \beta \in (\theta)\]
\end{claim}
\end{proof}
\end{theorem}




\end{document}