\documentclass[12pt]{article}
\usepackage{amsfonts, amsthm, amsmath}
\usepackage{verbatim}
\usepackage{enumerate}
\usepackage{graphicx}
\usepackage{amssymb}
\usepackage{epstopdf}
\usepackage{amsthm}
\usepackage{mathtools}
\usepackage{pb-diagram}
\usepackage{hyperref}
\usepackage{cancel}

\setlength{\textwidth}{6.5in}
\setlength{\oddsidemargin}{0in}
\setlength{\textheight}{9.5in}
\setlength{\topmargin}{0in}
\setlength{\headheight}{0in}
\setlength{\headsep}{0in}
\setlength{\parskip}{0pt}
\setlength{\parindent}{0pt}

\def\CC{\mathbb{C}}
\def\MM{\mathbb{M}}
\def\FF{\mathbb{F}}
\def\PP{\mathbb{P}}
\def\QQ{\mathbb{Q}}
\def\RR{\mathbb{R}}
\def\ZZ{\mathbb{Z}}
\def\gotha{\mathfrak{a}}
\def\gothb{\mathfrak{b}}
\def\gothm{\mathfrak{m}}
\def\gotho{\mathfrak{o}}
\def\gothp{\mathfrak{p}}
\def\gothq{\mathfrak{q}}
\DeclareMathOperator{\disc}{Disc}
\DeclareMathOperator{\Gal}{Gal}
\DeclareMathOperator{\GL}{GL}
\DeclareMathOperator{\Hom}{Hom}
\DeclareMathOperator{\Norm}{Norm}
\DeclareMathOperator{\Trace}{Trace}
\DeclareMathOperator{\Cl}{Cl}

\def\head#1{\medskip \noindent \textbf{#1}.}

\newtheorem{theorem}{Theorem}[section]
\newtheorem{lemma}[theorem]{Lemma}
\newtheorem{definition}{Definition}[section]
\newtheorem{example}{Example}[section]
\newtheorem{proposition}{Proposition}[section]
\newtheorem{corollary}{Corollary}[section]
\newtheorem*{note}{Note}
\newtheorem*{remark}{Remark}
\newtheorem*{claim}{Claim}


\begin{document}
\title{Probability \\ 3105}
\author{ Dr N Sidorova}
\date{Jan 2012}
\maketitle

\tableofcontents
\setcounter{tocdepth}{4}
\newpage

\section{Rigorous set up}

\begin{definition}[$\sigma$-algebra of sets]
Let $\Omega$ be a set and $\Sigma$ be a collection of sets. Then $\Sigma$ is a $\sigma$ - algebra if
\begin{enumerate}
\item $\Omega, \; \phi \in \Sigma$
\item $A \in \Sigma$ then $\Omega/A \in \Sigma$
\item $A_1, A_2, \dots \in \Sigma$ then $\cup_{1}^{\infty}A_i \in \Sigma$
\end{enumerate}
\end{definition}

\begin{definition}[Measure]
$\mu:\Sigma \rightarrow [0,\infty]$ is called a measure if
\begin{enumerate}
\item $\mu(\phi) = 0$
\item $A_1, A_2, \dots$ are disjoint then $\mu(\cup_{1}^{\infty}A_i) = \sum_{1}^{\infty}\mu(A_i)$
\end{enumerate}
\end{definition}

\begin{definition}[Probability Measure] 
A measure $\mu$ is called a Probability Measure denoted by $P$ if \[P(\Omega) =1\]
\end{definition}

\begin{definition}[Probability Space]
A triple $(\Omega, \Sigma, P)$, where $\Omega$ is a set, $\Sigma$ is a $\sigma$-algebra and $P$ is a probability measure, is called a Probability Space.  
\end{definition}

\begin{definition}[Measurable Function]
a function $X$ is called a Measurable function if 
\[\forall B \in \mathcal{B} \quad X^{-1}(B) = \{w\;:\; X(w) \in \mathcal{B} \} \in \Sigma \]
\end{definition}

\begin{definition}[Random Variable] 
A random variable is called a measurable function
\begin{align*}
X:\Omega &\rightarrow \RR\\
(\Omega, \Sigma) &\rightarrow (\RR , \underbrace{\mathcal{B}}_{\mathclap{borel \; \sigma -algebra}})
\end{align*}
\end{definition}

The idea:\\
$\Omega$ - Random outcomes\\
$\Sigma$ - All possible events\\
$P(E)$ - Probability of the event $E$

\begin{example}
Bernulli = "tossing a coin" = "$0$ or $1$ with probability $\frac{1}{2}$
\begin{align*}
&\Omega = \{H,T\}\\
&\Sigma = \{ \{H\},\{T\}, \{H,T\}, \phi \} = 2^{\Omega}\\
&P(\{H\}) = P(\{T\}) = \frac{1}{2}\\
&P(\{H,T\}) = 1\\
&P(\phi) = 0
\end{align*}
\begin{align*}
X:&H\rightarrow 1\\
&T\rightarrow 0
\end{align*}
\[\text{"Probability that $X=1$"} = P(\omega :X(\omega)=1) = P(\{H\}) = \frac{1}{2}\]
\end{example}

\begin{example}
Roll a die, spell the number, take $\#$ of letters
\begin{align*}
&\Omega = \{1,2,3,4,5,6\}\\
&\Sigma = 2^{\Omega} \quad \text{($\sigma$-algebra of all subsets)}\\
&P(\{1\})= \dots =P(\{6\}) = \frac{1}{6}\\
&P(\{1,3,5\}) = \frac{1}{6}+\frac{1}{6}+\frac{1}{6}= \frac{1}{2} \; etc \dots
\end{align*}
\begin{align*}
X:\quad &1\rightarrow 3\\
&2\rightarrow 3\\
&3\rightarrow 5\\
&4\rightarrow 4\\
&5\rightarrow 4\\
&6\rightarrow 3
\end{align*}
\end{example}

\begin{example}
Roll a die, spell the number, take $\#$ of letters but with the possibility of the dice rolling off the table and scoring $0$
\begin{align*}
&\Omega = \{1,2,3,4,5,6,0\}\\
&\Sigma = 2^{\Omega} \quad \text{($\sigma$-algebra of all subsets)}\\
&P(\{1\})= \dots =P(\{6\}) = \frac{1}{6} \quad 
P(\{0\}) = 0
\end{align*}
\begin{align*}
X:\quad  &0\rightarrow 4\\
&1\rightarrow 3\\
&2\rightarrow 3\\
&3\rightarrow 5\\
&4\rightarrow 4\\
&5\rightarrow 4\\
&6\rightarrow 3
\end{align*}
\end{example}

\begin{example}
Tossing a fair coin infinitely many times
\begin{align*}
&\Omega = [0,1]\\
&\Sigma = ?
\end{align*}
we can then represnt each event as a real number in $[0,1]$ for example all events where first three results are $HTH$ rest unknown:
\[\{\omega = 0.101***\} = \left[\frac{5}{8} , \frac{3}{4}\right] \]
All binary intervals must be in $\Sigma$. The minimal $\sigma$-algebra with this property is $\mathcal{B}$ the borel $\sigma$-algebra. 
\[P(\omega : 0.101*****\dots ) = leb\left[\frac{5}{8}, \frac{3}{4}\right] = \frac{1}{8}\]
\[\Rightarrow \text{P is a lebesque measure}\]
some questions that could be asked:\\
if $\omega = \omega_1, \omega_2 , \dots$
\begin{enumerate}[(a)]
\item what is the number in the $n^{th}$ position\\
$[0,1] \rightarrow \RR$\\
$0.\omega_1 \omega_2 \dots \mapsto  \omega_n$
\item how many $1$'s out of the first $n$ tosses?\\
$[0,1] \rightarrow \RR$\\
$0.\omega_1 \omega_2 \dots \mapsto \omega_1 + \dots + \omega_n$
\end{enumerate}
\end{example}

\begin{definition}
An event is an element of $\Sigma$
\end{definition}
Suppose an event $E$ occurs = suppose $\omega \in E$. An event occurs with probability $p= P(E)$. We are interested in: $P(X \in B) \equiv P(\omega \;:\;X(\omega) \subseteq B)$
\begin{definition}
\[\mu_x(B) := P(X\in B), \quad B \in \mathcal{B}\]
This probability measure on $(\RR , \mathcal{B})$ is called the distribution of $X$ or the law of $X$.
\end{definition}








\end{document}